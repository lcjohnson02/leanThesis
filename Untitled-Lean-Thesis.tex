% Options for packages loaded elsewhere
\PassOptionsToPackage{unicode}{hyperref}
\PassOptionsToPackage{hyphens}{url}
\PassOptionsToPackage{dvipsnames,svgnames,x11names}{xcolor}
%
\documentclass[
  letterpaper,
]{scrreprt}

\usepackage{amsmath,amssymb}
\usepackage{iftex}
\ifPDFTeX
  \usepackage[T1]{fontenc}
  \usepackage[utf8]{inputenc}
  \usepackage{textcomp} % provide euro and other symbols
\else % if luatex or xetex
  \usepackage{unicode-math}
  \defaultfontfeatures{Scale=MatchLowercase}
  \defaultfontfeatures[\rmfamily]{Ligatures=TeX,Scale=1}
\fi
\usepackage{lmodern}
\ifPDFTeX\else  
    % xetex/luatex font selection
  \setmonofont[]{JuliaMono}
\fi
% Use upquote if available, for straight quotes in verbatim environments
\IfFileExists{upquote.sty}{\usepackage{upquote}}{}
\IfFileExists{microtype.sty}{% use microtype if available
  \usepackage[]{microtype}
  \UseMicrotypeSet[protrusion]{basicmath} % disable protrusion for tt fonts
}{}
\makeatletter
\@ifundefined{KOMAClassName}{% if non-KOMA class
  \IfFileExists{parskip.sty}{%
    \usepackage{parskip}
  }{% else
    \setlength{\parindent}{0pt}
    \setlength{\parskip}{6pt plus 2pt minus 1pt}}
}{% if KOMA class
  \KOMAoptions{parskip=half}}
\makeatother
\usepackage{xcolor}
\setlength{\emergencystretch}{3em} % prevent overfull lines
\setcounter{secnumdepth}{0}
% Make \paragraph and \subparagraph free-standing
\ifx\paragraph\undefined\else
  \let\oldparagraph\paragraph
  \renewcommand{\paragraph}[1]{\oldparagraph{#1}\mbox{}}
\fi
\ifx\subparagraph\undefined\else
  \let\oldsubparagraph\subparagraph
  \renewcommand{\subparagraph}[1]{\oldsubparagraph{#1}\mbox{}}
\fi

\usepackage{color}
\usepackage{fancyvrb}
\newcommand{\VerbBar}{|}
\newcommand{\VERB}{\Verb[commandchars=\\\{\}]}
\DefineVerbatimEnvironment{Highlighting}{Verbatim}{commandchars=\\\{\}}
% Add ',fontsize=\small' for more characters per line
\usepackage{framed}
\definecolor{shadecolor}{RGB}{241,243,245}
\newenvironment{Shaded}{\begin{snugshade}}{\end{snugshade}}
\newcommand{\AlertTok}[1]{\textcolor[rgb]{0.68,0.00,0.00}{#1}}
\newcommand{\AnnotationTok}[1]{\textcolor[rgb]{0.37,0.37,0.37}{#1}}
\newcommand{\AttributeTok}[1]{\textcolor[rgb]{0.40,0.45,0.13}{#1}}
\newcommand{\BaseNTok}[1]{\textcolor[rgb]{0.68,0.00,0.00}{#1}}
\newcommand{\BuiltInTok}[1]{\textcolor[rgb]{0.00,0.23,0.31}{#1}}
\newcommand{\CharTok}[1]{\textcolor[rgb]{0.13,0.47,0.30}{#1}}
\newcommand{\CommentTok}[1]{\textcolor[rgb]{0.37,0.37,0.37}{#1}}
\newcommand{\CommentVarTok}[1]{\textcolor[rgb]{0.37,0.37,0.37}{\textit{#1}}}
\newcommand{\ConstantTok}[1]{\textcolor[rgb]{0.56,0.35,0.01}{#1}}
\newcommand{\ControlFlowTok}[1]{\textcolor[rgb]{0.00,0.23,0.31}{#1}}
\newcommand{\DataTypeTok}[1]{\textcolor[rgb]{0.68,0.00,0.00}{#1}}
\newcommand{\DecValTok}[1]{\textcolor[rgb]{0.68,0.00,0.00}{#1}}
\newcommand{\DocumentationTok}[1]{\textcolor[rgb]{0.37,0.37,0.37}{\textit{#1}}}
\newcommand{\ErrorTok}[1]{\textcolor[rgb]{0.68,0.00,0.00}{#1}}
\newcommand{\ExtensionTok}[1]{\textcolor[rgb]{0.00,0.23,0.31}{#1}}
\newcommand{\FloatTok}[1]{\textcolor[rgb]{0.68,0.00,0.00}{#1}}
\newcommand{\FunctionTok}[1]{\textcolor[rgb]{0.28,0.35,0.67}{#1}}
\newcommand{\ImportTok}[1]{\textcolor[rgb]{0.00,0.46,0.62}{#1}}
\newcommand{\InformationTok}[1]{\textcolor[rgb]{0.37,0.37,0.37}{#1}}
\newcommand{\KeywordTok}[1]{\textcolor[rgb]{0.00,0.23,0.31}{#1}}
\newcommand{\NormalTok}[1]{\textcolor[rgb]{0.00,0.23,0.31}{#1}}
\newcommand{\OperatorTok}[1]{\textcolor[rgb]{0.37,0.37,0.37}{#1}}
\newcommand{\OtherTok}[1]{\textcolor[rgb]{0.00,0.23,0.31}{#1}}
\newcommand{\PreprocessorTok}[1]{\textcolor[rgb]{0.68,0.00,0.00}{#1}}
\newcommand{\RegionMarkerTok}[1]{\textcolor[rgb]{0.00,0.23,0.31}{#1}}
\newcommand{\SpecialCharTok}[1]{\textcolor[rgb]{0.37,0.37,0.37}{#1}}
\newcommand{\SpecialStringTok}[1]{\textcolor[rgb]{0.13,0.47,0.30}{#1}}
\newcommand{\StringTok}[1]{\textcolor[rgb]{0.13,0.47,0.30}{#1}}
\newcommand{\VariableTok}[1]{\textcolor[rgb]{0.07,0.07,0.07}{#1}}
\newcommand{\VerbatimStringTok}[1]{\textcolor[rgb]{0.13,0.47,0.30}{#1}}
\newcommand{\WarningTok}[1]{\textcolor[rgb]{0.37,0.37,0.37}{\textit{#1}}}

\providecommand{\tightlist}{%
  \setlength{\itemsep}{0pt}\setlength{\parskip}{0pt}}\usepackage{longtable,booktabs,array}
\usepackage{calc} % for calculating minipage widths
% Correct order of tables after \paragraph or \subparagraph
\usepackage{etoolbox}
\makeatletter
\patchcmd\longtable{\par}{\if@noskipsec\mbox{}\fi\par}{}{}
\makeatother
% Allow footnotes in longtable head/foot
\IfFileExists{footnotehyper.sty}{\usepackage{footnotehyper}}{\usepackage{footnote}}
\makesavenoteenv{longtable}
\usepackage{graphicx}
\makeatletter
\def\maxwidth{\ifdim\Gin@nat@width>\linewidth\linewidth\else\Gin@nat@width\fi}
\def\maxheight{\ifdim\Gin@nat@height>\textheight\textheight\else\Gin@nat@height\fi}
\makeatother
% Scale images if necessary, so that they will not overflow the page
% margins by default, and it is still possible to overwrite the defaults
% using explicit options in \includegraphics[width, height, ...]{}
\setkeys{Gin}{width=\maxwidth,height=\maxheight,keepaspectratio}
% Set default figure placement to htbp
\makeatletter
\def\fps@figure{htbp}
\makeatother

\usepackage{mathrsfs}
\usepackage{twemojis}
\usepackage[normalem]{ulem}
\usepackage{scrlayer-scrpage}
\usepackage{textpos}
\automark{section}
\newcommand{\copyrightnotice}{\footnotesize\ Logan Johnson}
\newcommand{\longcopyrightnotice}{\footnotesize\ Logan Johnson 2023}
\setkomafont{pagefoot}{\upshape}
\ifoot*{\longcopyrightnotice}
\cfoot*{}
\ofoot*{\pagemark}

\makeatletter
\def\redsquiggly{\bgroup \markoverwith{\textcolor{red}{\lower3.5\p@\hbox{\sixly \char58}}}\ULon}
\def\brownsquiggly{\bgroup \markoverwith{\textcolor[HTML]{B8860B}{\lower3.5\p@\hbox{\sixly \char58}}}\ULon}
\def\bluesquiggly{\bgroup \markoverwith{\textcolor[HTML]{1E90FF}{\lower3.5\p@\hbox{\sixly \char58}}}\ULon}
\makeatother

\renewcommand{\NormalTok}[1]{\textcolor[HTML]{000000}{#1}}
\renewcommand{\KeywordTok}[1]{\textcolor[HTML]{0000FF}{#1}}
\renewcommand{\SpecialCharTok}[1]{}
\renewcommand{\ErrorTok}[1]{\redsquiggly{#1}}
\renewcommand{\WarningTok}[1]{\redsquiggly{\textcolor[HTML]{0000FF}{#1}}}
\renewcommand{\StringTok}[1]{\textcolor[HTML]{A52A2A}{#1}}
\renewcommand{\CommentTok}[1]{\textcolor[HTML]{008000}{#1}}
\renewcommand{\InformationTok}[1]{\textcolor[HTML]{D2691E}{\textbf{#1}}}
\renewcommand{\RegionMarkerTok}[1]{▼\:\textcolor[HTML]{008000}{\textbf{#1}}}
\renewcommand{\SpecialStringTok}[1]{\textcolor[HTML]{4682B4}{\textbf{#1}}}
\renewcommand{\ConstantTok}[1]{\textcolor[HTML]{DC143C}{#1}}
\renewcommand{\AnnotationTok}[1]{\brownsquiggly{#1}}
\renewcommand{\AlertTok}[1]{\brownsquiggly{\textcolor[HTML]{0000FF}{#1}}}
\renewcommand{\OtherTok}[1]{\bluesquiggly{#1}}
\renewcommand{\DocumentationTok}[1]{\bluesquiggly{\textcolor[HTML]{0000FF}{#1}}}

\newenvironment{ind}
	{\begin{list}{}{\setlength{\leftmargin}{1em}}\item\relax}
	{\end{list}}

%redefines Shaded so it can't break
\newcommand{\nobreakShaded}{\renewenvironment{Shaded}
	{\begin{tcolorbox}[frame hidden, enhanced, interior hidden, boxrule=0pt,
		borderline west={3pt}{0pt}{shadecolor}, sharp corners]}
	{\end{tcolorbox}}}

%Make end of environment ignore pars that come after it
\def\useignorespacesandallpars#1\ignorespaces\fi{%
#1\fi\ignorespacesandallpars}

\makeatletter
\def\ignorespacesandallpars{%
  \@ifnextchar\par
    {\expandafter\ignorespacesandallpars\@gobble}%
    {}%
}
\makeatother

\newenvironment{inpt}
	{\nobreakShaded\noindent\begin{minipage}[t]{0.63\textwidth}
		\uline{Lean File}}
	{\end{minipage}\hfill\useignorespacesandallpars}

\newenvironment{outpt}
	{\nobreakShaded\begin{minipage}[t]{0.32\textwidth}
		\uline{Tactic State in Infoview}}
	{\end{minipage}}

\newenvironment{bef}
	{\nobreakShaded\noindent\begin{minipage}[t]{0.475\textwidth}
		\uline{Tactic State Before Using Strategy}}
	{\end{minipage}\hfill\useignorespacesandallpars}

\newenvironment{aft}
	{\nobreakShaded\begin{minipage}[t]{0.475\textwidth}
		\uline{Tactic State After Using Strategy}}
	{\end{minipage}}

\newenvironment{numex}[1]
	{\begin{minipage}[t]{0.04\textwidth}\vspace{8pt}{#1}.
		\end{minipage}\nobreakShaded\begin{minipage}[t]{0.96\textwidth}\vspace{0pt}}
	{\end{minipage}}

\newenvironment{mdsk}
	{\medskip}
	{}

\newenvironment{absnobreak}
  {\par\nobreak\vfil\penalty0\vfilneg
   \vtop\bgroup}
  {\par\xdef\tpd{\the\prevdepth}\egroup
   \prevdepth=\tpd}

\newcommand{\excl}[1]{}
\newcommand{\incl}[1]{#1}

\newcommand{\setmin}{\mathbin{\backslash}}
\newcommand{\symmdiff}{\bigtriangleup}

\pagenumbering{roman}  %So front matter uses roman numerals.  Switch back to arabic at beginning of preface.
\publishers{\longcopyrightnotice}
\makeatletter
\makeatother
\makeatletter
\@ifpackageloaded{bookmark}{}{\usepackage{bookmark}}
\makeatother
\makeatletter
\@ifpackageloaded{caption}{}{\usepackage{caption}}
\AtBeginDocument{%
\ifdefined\contentsname
  \renewcommand*\contentsname{Table of contents}
\else
  \newcommand\contentsname{Table of contents}
\fi
\ifdefined\listfigurename
  \renewcommand*\listfigurename{List of Figures}
\else
  \newcommand\listfigurename{List of Figures}
\fi
\ifdefined\listtablename
  \renewcommand*\listtablename{List of Tables}
\else
  \newcommand\listtablename{List of Tables}
\fi
\ifdefined\figurename
  \renewcommand*\figurename{Figure}
\else
  \newcommand\figurename{Figure}
\fi
\ifdefined\tablename
  \renewcommand*\tablename{Table}
\else
  \newcommand\tablename{Table}
\fi
}
\@ifpackageloaded{float}{}{\usepackage{float}}
\floatstyle{ruled}
\@ifundefined{c@chapter}{\newfloat{codelisting}{h}{lop}}{\newfloat{codelisting}{h}{lop}[chapter]}
\floatname{codelisting}{Listing}
\newcommand*\listoflistings{\listof{codelisting}{List of Listings}}
\usepackage{amsthm}
\theoremstyle{remark}
\AtBeginDocument{\renewcommand*{\proofname}{Proof}}
\newtheorem*{remark}{Remark}
\newtheorem*{solution}{Solution}
\makeatother
\makeatletter
\@ifpackageloaded{caption}{}{\usepackage{caption}}
\@ifpackageloaded{subcaption}{}{\usepackage{subcaption}}
\makeatother
\makeatletter
\@ifpackageloaded{tcolorbox}{}{\usepackage[skins,breakable]{tcolorbox}}
\makeatother
\makeatletter
\@ifundefined{shadecolor}{\definecolor{shadecolor}{rgb}{.97, .97, .97}}
\makeatother
\makeatletter
\makeatother
\makeatletter
\makeatother
\ifLuaTeX
  \usepackage{selnolig}  % disable illegal ligatures
\fi
\IfFileExists{bookmark.sty}{\usepackage{bookmark}}{\usepackage{hyperref}}
\IfFileExists{xurl.sty}{\usepackage{xurl}}{} % add URL line breaks if available
\urlstyle{same} % disable monospaced font for URLs
\hypersetup{
  pdftitle={Untitled Lean Thesis},
  pdfauthor={Logan Johnson},
  colorlinks=true,
  linkcolor={blue},
  filecolor={Maroon},
  citecolor={Blue},
  urlcolor={Blue},
  pdfcreator={LaTeX via pandoc}}

\title{Untitled Lean Thesis}
\author{Logan Johnson}
\date{Invalid Date}

\begin{document}
\newcommand{\fromtop}[1]{%
  \dimexpr-1in-\topskip-\topmargin-\headheight-\headsep+#1\relax
}
\newcommand{\fromleft}[1]{%
  \dimexpr-1in-\oddsidemargin+#1\relax
}
\setlength{\textwidth}{6in}
\begin{titlepage}
\cleardoublepage
\thispagestyle{empty}
\begin{textblock*}{\paperwidth}(\fromleft{0cm},\fromtop{144pt})
\centering
UNTITLED LEAN THESIS
\end{textblock*}
\begin{textblock*}{\paperwidth}(\fromleft{0cm},\fromtop{300pt})
\centering
A Thesis Submitted to the Faculty of\\
\vspace{12pt}
Georgetown College\\
\vspace{12pt}
In Partial Fulfillment of the Requirements for the\\
\vspace{12pt}
Honors Program
\end{textblock*}
\begin{textblock*}{\paperwidth}(\fromleft{0cm},\fromtop{492pt})
\centering
By\\
\vspace{12pt}
Logan Johnson
\end{textblock*}
\begin{textblock*}{\paperwidth}(\fromleft{0cm},\fromtop{600pt})
\centering
Georgetown, Kentucky\\
\vspace{12pt}
May 2024
\end{textblock*}
\pagebreak
\thispagestyle{empty}
\begin{textblock*}{\paperwidth}(\fromleft{0cm},\fromtop{180pt})
\centering
Abstract\\
\vspace{12pt}
UNTITLED LEAN THESIS\\
\vspace{12pt}
Logan C. Johnson\\
\vspace{12pt}
idk dr burch?? maybe dr white???
\end{textblock*}
\begin{textblock*}{\textwidth}(\fromleft{1.5in},\fromtop{300pt})
\raggedright
Here is the text of your abstract. It goes on and on and on. It goes on like this for about
150 words, so it should all fit on this page. Note that the Abstract comes before the title page and
has no page number. The rest of this paragraph is a filler. It goes on like this for about 150
words, so it should all fit on this page. Note that the Abstract comes before the title page and has
no page number. It goes on like this for about 150 words, so it should all fit on this page. Note
that the Abstract comes before the title page and has no page number. It goes on like this for
about 150 words, so it should all fit on this page. Note that the Abstract comes before the title
page and has no page number. It goes on like this for about 150 words, so it should all fit on this
page. Note that the Abstract comes before the title page and has no page number. It goes on like
this for about 150 words, so it should all fit on this page. Note that the Abstract comes before the
title page and has no page number. It goes on like this for about 150 words, so it should all fit on
this page. Note that the Abstract comes before the title page and has no page number. If your
abstract is more than 250 words, consider shortening it.
\end{textblock*}
\pagebreak
\thispagestyle{empty}
\begin{textblock*}{\paperwidth}(\fromleft{2in},\fromtop{144pt})
APPROVED BY THE DIRECTOR OF HONORS THESES:
\end{textblock*}
\begin{textblock*}{\paperwidth}(\fromleft{3in},\fromtop{192pt})
\underline{\hspace{4.5in}}
\end{textblock*}
\begin{textblock*}{\paperwidth}(\fromleft{3in},\fromtop{216pt})
Dr. Homer White, Department of Mathematics
\end{textblock*}
\begin{textblock*}{\paperwidth}(\fromleft{2in},\fromtop{396pt})
APPROVED BY THE HONORS PROGRAM:
\end{textblock*}
\begin{textblock*}{\paperwidth}(\fromleft{2in},\fromtop{444pt})
\underline{\hspace{4.5in}}
\end{textblock*}
\begin{textblock*}{\paperwidth}(\fromleft{2in},\fromtop{468pt})
Dr. Barbara Burch, Director
\end{textblock*}
\begin{textblock*}{\paperwidth}(\fromleft{1.5in},\fromtop{612pt})
DATE:\underline{\hspace{2in}}
\end{textblock*}
\end{titlepage}%Can't be in preamble because Quarto loads amsthm too late.
\theoremstyle{plain}
\newtheorem*{thm}{Theorem}
\newcommand{\thmnm}{Theorem}
\newtheorem*{namedthm}{\thmnm}
\theoremstyle{definition}
\newtheorem*{dfn}{Definition}
\newcommand{\defnm}{Definition}
\newtheorem*{nameddfn}{\defnm}

\newenvironment{nthm}[1]
  {\renewcommand{\thmnm}{#1}\begin{namedthm}}
  {\end{namedthm}}

\newenvironment{ndfn}[1]
  {\renewcommand{\defnm}{#1}\begin{nameddfn}}
  {\end{nameddfn}}

\newenvironment{npf}[1]
  {\begin{proof}[#1]}
  {\end{proof}}

% Usage:
% ::: {.thm} will create an unnumbered thm environment with title Theorem.
% ::: {.nthm arguments="Name of Theorem"} will create an unnumbered theorem whose title is Name of Theorem.

\let\oldgreater\textgreater
\renewcommand{\textgreater}{\null\oldgreater}   % To prevent => changing to double arrow

\ifdefined\Shaded\renewenvironment{Shaded}{\begin{tcolorbox}[breakable, enhanced, boxrule=0pt, interior hidden, frame hidden, borderline west={3pt}{0pt}{shadecolor}, sharp corners]}{\end{tcolorbox}}\fi

\renewcommand*\contentsname{Table of contents}
{
\hypersetup{linkcolor=}
\setcounter{tocdepth}{1}
\tableofcontents
}
\bookmarksetup{startatroot}

\hypertarget{preface}{%
\chapter*{Preface}\label{preface}}
\addcontentsline{toc}{chapter}{Preface}

\markboth{Preface}{Preface}

\pagenumbering{arabic}

\bookmarksetup{startatroot}

\hypertarget{real-analysis}{%
\chapter{Real Analysis}\label{real-analysis}}

\bookmarksetup{startatroot}

\hypertarget{functional-programming}{%
\chapter{Functional Programming}\label{functional-programming}}

\bookmarksetup{startatroot}

\hypertarget{lean-as-a-theorem-prover}{%
\chapter{Lean as a Theorem Prover}\label{lean-as-a-theorem-prover}}

\hypertarget{differences-from-paragraph-style-proofs}{%
\section{Differences From Paragraph Style
Proofs}\label{differences-from-paragraph-style-proofs}}

Despite the incredible power that lean could provide in the verification
of mathematical proofs, this does pose some difficulties, namely the
ease with which the aforementioned proofs can be written up. Typically,
proofs are simply written up in a paragraph style, where the steps being
taken and the theorems being applied are laid out in plain terms so that
it can be easily understood by fellow mathematicians. There are often
times when mathematicians will take things for granted or skip over
steps that they think the reader will either already know to be fact or
can easily reason out for themselves when writing out typical proofs.
This lax approach for conveying information simply does not work when
trying to communicate with technology, and a much more specific and
methodical approach must be adopted in order to take advantage of the
logical verification benefits. Thankfully, lean has a community working
to create libraries of previously proven theorems that can be applied to
speed up the writing and verification of future proofs. This thankfully
means that all proofs do not need to be taken all the way back to basic
axioms: Users can save time by avoiding proving adjacent theorems and
instead focus only on the immediately relevant steps of their proof.

For each of the following proofs, I will first provide a typical
``paragraph style'' version of the proof, so the differences between the
two can easily be compared.

\hypertarget{inequality-addition}{%
\section{Inequality Addition}\label{inequality-addition}}

\hypertarget{paragraph-style-proof}{%
\subsection{Paragraph Style Proof}\label{paragraph-style-proof}}

\begin{thm}
If \(a < b\) and \(c \le d\), prove that \(a + c < b + d\)

\end{thm}

There are multiple ways to approach this in a paragraph style proof, so
I will attempt to have this proof follow along the same lines as the
lean proof.

\begin{proof}

There are two possible cases: either \(c = d\) or \(c < d\). We will
first consider the case where \(c = d\). We know \(a < b\), so it would
also be true that \(a + c < b + c\). Then because \(c = d\),
\(a + c < b + d\). Now consider the case where \(c < d\). We know
\(a < b\), so \(a + c < b + c\) and \(b + c < b + d\) because \(c < d\).
Thus by transitivity of inequalities, we could say \(a + c < b + d\)
\excl{~□}\qedhere

\end{proof}

\hypertarget{lean-proof}{%
\subsection{Lean Proof}\label{lean-proof}}

\hypertarget{seting-up-the-problem}{%
\subsubsection{Seting up the problem}\label{seting-up-the-problem}}

Here I put the theorem we want to prove into lean and we can see the
resulting infoview panel. I name our two assumptions \texttt{h1} and
\texttt{h2}, for hypotheses one and two. After a colon I then write out
the thing I am trying to prove with those hypotheses and use \texttt{by}
to put lean into tactic mode.

It can now be seen that the infoview panel lists out both of our
hypotheses as well as the goal we are working towards at the bottom.
This panel will continue to change as more code is added to the lean
file.

\begin{inpt}

\begin{Shaded}
\begin{Highlighting}[]
\KeywordTok{example}\NormalTok{ (a b c d : ℝ) (h1: a \textless{} b) }
\NormalTok{    (h2 : c ≤ d) : a + c \textless{} b + d := }\KeywordTok{by}
  
  \SpecialCharTok{**}\WarningTok{done}\SpecialCharTok{::}
\end{Highlighting}
\end{Shaded}

\end{inpt}

\begin{outpt}

\begin{Shaded}
\begin{Highlighting}[]
\InformationTok{R}\NormalTok{: Type u\_1}
\InformationTok{inst✝}\NormalTok{: Ring R}
\InformationTok{abcd}\NormalTok{: ℝ}
\InformationTok{h1}\NormalTok{: a \textless{} b}
\InformationTok{h2}\NormalTok{: c ≤ d}
\NormalTok{⊢ a + c \textless{} b + d}
\end{Highlighting}
\end{Shaded}

\end{outpt}

\hypertarget{step-1}{%
\subsubsection{Step 1}\label{step-1}}

Here I lay out the two possible cases of our second hypothesis which
allows me to strengthen the information that we know. We see this
strengthened hypothesis reflected in \texttt{h3} in the infoview.

\begin{inpt}

\begin{Shaded}
\begin{Highlighting}[]
\KeywordTok{example}\NormalTok{ (a b c d : ℝ) (h1: a \textless{} b) }
\NormalTok{    (h2 : c ≤ d) : a + c \textless{} b + d := }\KeywordTok{by}
  \KeywordTok{by\_cases}\NormalTok{ h3 : c = d}

  \SpecialCharTok{**}\WarningTok{done}\SpecialCharTok{::}
\end{Highlighting}
\end{Shaded}

\end{inpt}

\begin{outpt}

\begin{Shaded}
\begin{Highlighting}[]
\InformationTok{R}\NormalTok{: Type u\_1}
\InformationTok{inst✝}\NormalTok{: Ring R}
\InformationTok{abcd}\NormalTok{: ℝ}
\InformationTok{h1}\NormalTok{: a \textless{} b}
\InformationTok{h2}\NormalTok{: c ≤ d}
\InformationTok{h3}\NormalTok{: c = d}
\NormalTok{⊢ a + c \textless{} b + d}
\end{Highlighting}
\end{Shaded}

\end{outpt}

\hypertarget{step-2}{%
\subsubsection{Step 2}\label{step-2}}

Here I used hypothesis 3 to rewrite he c in our final goal as a d.~This
change is reflected in the infoview for this step.

\begin{inpt}

\begin{Shaded}
\begin{Highlighting}[]
\KeywordTok{example}\NormalTok{ (a b c d : ℝ) (h1: a \textless{} b) }
\NormalTok{    (h2 : c ≤ d) : a + c \textless{} b + d := }\KeywordTok{by}
  \KeywordTok{by\_cases}\NormalTok{ h3 : c = d}
  \KeywordTok{rw}\NormalTok{ [h3]}

  \SpecialCharTok{**}\WarningTok{done}\SpecialCharTok{::}
\end{Highlighting}
\end{Shaded}

\end{inpt}

\begin{outpt}

\begin{Shaded}
\begin{Highlighting}[]
\InformationTok{R}\NormalTok{: Type u\_1}
\InformationTok{inst✝}\NormalTok{: Ring R}
\InformationTok{abcd}\NormalTok{: ℝ}
\InformationTok{h1}\NormalTok{: a \textless{} b}
\InformationTok{h2}\NormalTok{: c ≤ d}
\InformationTok{h3}\NormalTok{: c = d}
\NormalTok{⊢ a + d \textless{} b + d}
\end{Highlighting}
\end{Shaded}

\end{outpt}

\hypertarget{step-3}{%
\subsubsection{Step 3}\label{step-3}}

In this step I applied a theorem already in the Mathlib library for
lean. The \texttt{add\_lt\_add\_right} theorem simply states that if you
have a \texttt{b\ \textless{}\ c}, then
\texttt{b\ +\ a\ \textless{}\ c\ +\ a} which is exactly what we need to
prove the goal for the first case. As the first case has been completed,
the infoview then switches to the second case which is reflected in the
new \texttt{h3} and reset goal.

\begin{inpt}

\begin{Shaded}
\begin{Highlighting}[]
\KeywordTok{example}\NormalTok{ (a b c d : ℝ) (h1: a \textless{} b) }
\NormalTok{    (h2 : c ≤ d) : a + c \textless{} b + d := }\KeywordTok{by}
  \KeywordTok{by\_cases}\NormalTok{ h3 : c = d}
  \KeywordTok{rw}\NormalTok{ [h3]}
  \KeywordTok{apply}\NormalTok{ add\_lt\_add\_right h1}

  \SpecialCharTok{**}\WarningTok{done}\SpecialCharTok{::}
\end{Highlighting}
\end{Shaded}

\end{inpt}

\begin{outpt}

\begin{Shaded}
\begin{Highlighting}[]
\InformationTok{R}\NormalTok{: Type u\_1}
\InformationTok{inst✝}\NormalTok{: Ring R}
\InformationTok{abcd}\NormalTok{: ℝ}
\InformationTok{h1}\NormalTok{: a \textless{} b}
\InformationTok{h2}\NormalTok{: c ≤ d}
\InformationTok{h3}\NormalTok{: ¬c = d}
\NormalTok{⊢ a + c \textless{} b + d}
\end{Highlighting}
\end{Shaded}

\end{outpt}

\hypertarget{step-4}{%
\subsubsection{Step 4}\label{step-4}}

In order to better work with our new hypothesis, I use a tactic which
pushes the negation symbol further into the thing it is negating. This
results in a hypothesis which can actually be applied later on.

\begin{inpt}

\begin{Shaded}
\begin{Highlighting}[]
\KeywordTok{example}\NormalTok{ (a b c d : ℝ) (h1: a \textless{} b) }
\NormalTok{    (h2 : c ≤ d) : a + c \textless{} b + d := }\KeywordTok{by}
  \KeywordTok{by\_cases}\NormalTok{ h3 : c = d}
  \KeywordTok{rw}\NormalTok{ [h3]}
  \KeywordTok{apply}\NormalTok{ add\_lt\_add\_right h1}
  \KeywordTok{push\_neg} \KeywordTok{at}\NormalTok{ h3}

  \SpecialCharTok{**}\WarningTok{done}\SpecialCharTok{::}
\end{Highlighting}
\end{Shaded}

\end{inpt}

\begin{outpt}

\begin{Shaded}
\begin{Highlighting}[]
\InformationTok{R}\NormalTok{: Type u\_1}
\InformationTok{inst✝}\NormalTok{: Ring R}
\InformationTok{abcd}\NormalTok{: ℝ}
\InformationTok{h1}\NormalTok{: a \textless{} b}
\InformationTok{h2}\NormalTok{: c ≤ d}
\InformationTok{h3}\NormalTok{: c ≠ d}
\NormalTok{⊢ a + c \textless{} b + d}
\end{Highlighting}
\end{Shaded}

\end{outpt}

\hypertarget{step-5}{%
\subsubsection{Step 5}\label{step-5}}

Here I am laying out a new hypothesis which will be useful later in the
proof. This hypothesis seems like an obvious conclusion based on
hypotheses two and three, but we must still lay it out simply for lean
if we want to actually use it. The infoview panel always displays the
most current goal, which is why it is displaying the goal for
\texttt{h4} rather than the main goal.

\begin{inpt}

\begin{Shaded}
\begin{Highlighting}[]
\KeywordTok{example}\NormalTok{ (a b c d : ℝ) (h1: a \textless{} b) }
\NormalTok{    (h2 : c ≤ d) : a + c \textless{} b + d := }\KeywordTok{by}
  \KeywordTok{by\_cases}\NormalTok{ h3 : c = d}
  \KeywordTok{rw}\NormalTok{ [h3]}
  \KeywordTok{apply}\NormalTok{ add\_lt\_add\_right h1}
  \KeywordTok{push\_neg} \KeywordTok{at}\NormalTok{ h3}
  \KeywordTok{have}\NormalTok{ h4 : c \textless{} d := }\KeywordTok{by}

  \SpecialCharTok{**}\WarningTok{done}\SpecialCharTok{::}
\end{Highlighting}
\end{Shaded}

\end{inpt}

\begin{outpt}

\begin{Shaded}
\begin{Highlighting}[]
\InformationTok{R}\NormalTok{: Type u\_1}
\InformationTok{inst✝}\NormalTok{: Ring R}
\InformationTok{abcd}\NormalTok{: ℝ}
\InformationTok{h1}\NormalTok{: a \textless{} b}
\InformationTok{h2}\NormalTok{: c ≤ d}
\InformationTok{h3}\NormalTok{: c ≠ d}
\NormalTok{⊢ c \textless{} d}
\end{Highlighting}
\end{Shaded}

\end{outpt}

\hypertarget{step-6}{%
\subsubsection{Step 6}\label{step-6}}

Here I apply another theorem already in lean which takes the information
\texttt{h3} and \texttt{h2} gives us and shows our current goal. Writing
out \texttt{h4} like this is technically optional, as lean allows you to
evaluate tactics within arguments for other tactics. Despite this, I
personally find it more convenient and clear to write out extra
hypotheses like this rather than just giving the body of the argument
when necessary. Now that our new hypothesis has been proven, the
infoview displays that we have no goals until we get back into our main
theorem.

\begin{inpt}

\begin{Shaded}
\begin{Highlighting}[]
\KeywordTok{example}\NormalTok{ (a b c d : ℝ) (h1: a \textless{} b) }
\NormalTok{    (h2 : c ≤ d) : a + c \textless{} b + d := }\KeywordTok{by}
  \KeywordTok{by\_cases}\NormalTok{ h3 : c = d}
  \KeywordTok{rw}\NormalTok{ [h3]}
  \KeywordTok{apply}\NormalTok{ add\_lt\_add\_right h1}
  \KeywordTok{push\_neg} \KeywordTok{at}\NormalTok{ h3}
  \KeywordTok{have}\NormalTok{ h4 : c \textless{} d := }\KeywordTok{by}
    \KeywordTok{apply}\NormalTok{ Ne.lt\_of\_le h3 h2}

  \SpecialCharTok{**}\WarningTok{done}\SpecialCharTok{::}
\end{Highlighting}
\end{Shaded}

\end{inpt}

\begin{outpt}

\begin{Shaded}
\begin{Highlighting}[]
\SpecialStringTok{No}\InformationTok{ }\SpecialStringTok{goals}
\end{Highlighting}
\end{Shaded}

\end{outpt}

\hypertarget{step-7}{%
\subsubsection{Step 7}\label{step-7}}

I now use the calc tactic to work through the rest of the theorem. This
tactic is quite useful as it allows us to chain together multiple
equalities or inequalities while still giving proofs for each step. This
is essentially a shortcut of writing out individual hypotheses and then
using the rewrite tactic to get our desired goal.

In this case, I only need to do two steps of chaining inequalities,
where I use transitivity to show that the starting value is less than
the final value. It essentially follows the same path as the paragraph
style proof, where the tactics \texttt{add\_lt\_add\_right} and
\texttt{add\_lt\_add\_left} justify the steps taken.

\begin{inpt}

\begin{Shaded}
\begin{Highlighting}[]
\KeywordTok{example}\NormalTok{ (a b c d : ℝ) (h1: a \textless{} b) }
\NormalTok{    (h2 : c ≤ d) : a + c \textless{} b + d := }\KeywordTok{by}
  \KeywordTok{by\_cases}\NormalTok{ h3 : c = d}
  \KeywordTok{rw}\NormalTok{ [h3]}
  \KeywordTok{apply}\NormalTok{ add\_lt\_add\_right h1}
  \KeywordTok{push\_neg} \KeywordTok{at}\NormalTok{ h3}
  \KeywordTok{have}\NormalTok{ h4 : c \textless{} d := }\KeywordTok{by}
    \KeywordTok{apply}\NormalTok{ Ne.lt\_of\_le h3 h2}
  \KeywordTok{exact} \KeywordTok{calc}
\NormalTok{    a + c \textless{} b + c := add\_lt\_add\_right h1 c}
\NormalTok{    \_ \textless{} b + d := add\_lt\_add\_left h4 b}
  \KeywordTok{done}
\end{Highlighting}
\end{Shaded}

\end{inpt}

\begin{outpt}

\begin{Shaded}
\begin{Highlighting}[]
\SpecialStringTok{No}\InformationTok{ }\SpecialStringTok{goals}
\end{Highlighting}
\end{Shaded}

\end{outpt}

\hypertarget{a-is-less-than-or-equal-to-b}{%
\section{a is Less Than or Equal to
b}\label{a-is-less-than-or-equal-to-b}}

\hypertarget{paragraph-style-proof-1}{%
\subsection{Paragraph Style Proof}\label{paragraph-style-proof-1}}

\begin{thm}
Suppose that \(a, b \in \mathbb{R}\) and for every \(\varepsilon > 0\),
we have \(a \le b + \varepsilon\). Show that \(a \le b\).

\end{thm}

\begin{proof}

Assume for the sake of contradiction that \(a\) is not less than or
equal to \(b\). Then it would be true that \(a > b\). Now consider the
case where \(\varepsilon = \frac{a - b}{2}.\) Then since \(a > b\),
epsilon is positive and by our assumption then
\(a \le b + \varepsilon\). Then \begin{align*}
a & \le b + \varepsilon \\
& = b + \frac{a - b}{2} \\
& = b + \frac{a}{2} - \frac{b}{2} \\
& = \frac{a}{2} + \frac{b}{2}. 
\end{align*} So now, \begin{align*}
a & \le \frac{a}{2} + \frac{b}{2} \\
a - \frac{a}{2} & \le \frac{b}{2} \\
\frac{a}{2} & \le \frac{b}{2} \\
a & \le b.
\end{align*} But now we have that \(a \le b\) and \(a > b\), a
contradiction! \excl{~□}\qedhere

\end{proof}

\hypertarget{lean-proof-1}{%
\subsection{Lean Proof}\label{lean-proof-1}}

\hypertarget{setting-up-the-problem}{%
\subsubsection{Setting up the problem}\label{setting-up-the-problem}}

I again set up the proof with our one hypothesis and the goal we want to
prove. These are then seen listed in the infoview on the right.

\begin{inpt}

\begin{Shaded}
\begin{Highlighting}[]
\KeywordTok{example}\NormalTok{ (a b : ℝ) (h1 : ∀ ε : ℝ, }
\NormalTok{    ε \textgreater{} 0 → a ≤ b + ε) : }
\NormalTok{    a ≤ b := }\KeywordTok{by}

  \SpecialCharTok{**}\WarningTok{done}\SpecialCharTok{::}
\end{Highlighting}
\end{Shaded}

\end{inpt}

\begin{outpt}

\begin{Shaded}
\begin{Highlighting}[]
\InformationTok{R}\NormalTok{: Type u\_1}
\InformationTok{inst✝}\NormalTok{: Ring R}
\InformationTok{ab}\NormalTok{: ℝ}
\InformationTok{h1}\NormalTok{: ∀ (ε : ℝ), ε \textgreater{} 0 → }
\SpecialCharTok{\textgreater{}\textgreater{}}\NormalTok{  a ≤ b + ε}
\NormalTok{⊢ a ≤ b}
\end{Highlighting}
\end{Shaded}

\end{outpt}

\hypertarget{step-1-1}{%
\subsubsection{Step 1}\label{step-1-1}}

The \texttt{by\_contra} tactic allows me to complete this problem using
proof by contradiction. This tactic automatically creates a hypothesis
containing the negation of the final goal, I named it \texttt{h2}, and
changes the final goal to \texttt{False} meaning that it needs a
contradiction.

\begin{inpt}

\begin{Shaded}
\begin{Highlighting}[]
\KeywordTok{example}\NormalTok{ (a b : ℝ) (h1 : ∀ ε : ℝ, }
\NormalTok{    ε \textgreater{} 0 → a ≤ b + ε) : }
\NormalTok{    a ≤ b := }\KeywordTok{by}
  \KeywordTok{by\_contra}\NormalTok{ h2}

  \SpecialCharTok{**}\WarningTok{done}\SpecialCharTok{::}
\end{Highlighting}
\end{Shaded}

\end{inpt}

\begin{outpt}

\begin{Shaded}
\begin{Highlighting}[]
\InformationTok{R}\NormalTok{: Type u\_1}
\InformationTok{inst✝}\NormalTok{: Ring R}
\InformationTok{ab}\NormalTok{: ℝ}
\InformationTok{h1}\NormalTok{: ∀ (ε : ℝ), ε \textgreater{} 0 → }
\SpecialCharTok{\textgreater{}\textgreater{}}\NormalTok{  a ≤ b + ε}
\InformationTok{h2}\NormalTok{: ¬a ≤ b}
\NormalTok{⊢ False}
\end{Highlighting}
\end{Shaded}

\end{outpt}

\hypertarget{step-2-1}{%
\subsubsection{Step 2}\label{step-2-1}}

Here I use the \texttt{push\_neg} tactic similarly to the previous
example to het a usable version of \texttt{h2} as well as pick a
specific epsilon for which we will find a contradiction. This new
epsilon will now show up in the infoview in the side and can be used in
our problem.

\begin{inpt}

\begin{Shaded}
\begin{Highlighting}[]
\KeywordTok{example}\NormalTok{ (a b : ℝ) (h1 : ∀ ε : ℝ, }
\NormalTok{    ε \textgreater{} 0 → a ≤ b + ε) : }
\NormalTok{    a ≤ b := }\KeywordTok{by}
  \KeywordTok{by\_contra}\NormalTok{ h2}
  \KeywordTok{push\_neg} \KeywordTok{at}\NormalTok{ h2}
  \KeywordTok{let}\NormalTok{ ε := (a {-} b) / 2}

  \SpecialCharTok{**}\WarningTok{done}\SpecialCharTok{::}
\end{Highlighting}
\end{Shaded}

\end{inpt}

\begin{outpt}

\begin{Shaded}
\begin{Highlighting}[]
\InformationTok{R}\NormalTok{: Type u\_1}
\InformationTok{inst✝}\NormalTok{: Ring R}
\InformationTok{ab}\NormalTok{: ℝ}
\InformationTok{h1}\NormalTok{: ∀ (ε : ℝ), ε \textgreater{} 0 → }
\SpecialCharTok{\textgreater{}\textgreater{}}\NormalTok{  a ≤ b + ε}
\InformationTok{h2}\NormalTok{: b \textless{} a}
\InformationTok{ε}\NormalTok{: ℝ := (a {-} b) / 2}
\NormalTok{⊢ False}
\end{Highlighting}
\end{Shaded}

\end{outpt}

\hypertarget{step-3-1}{%
\subsubsection{Step 3}\label{step-3-1}}

Here I lay out a hypothesis that we will later be able to apply to
\texttt{h1}. Saying that epsilon was positive in the paragraph style
proof fairly simpler, where we only really need to justify that
\(a - b\) is positive. In lean however, it requires a bit more effort
and as such I put in its own hypothesis.

\begin{inpt}

\begin{Shaded}
\begin{Highlighting}[]
\KeywordTok{example}\NormalTok{ (a b : ℝ) (h1 : ∀ ε : ℝ, }
\NormalTok{    ε \textgreater{} 0 → a ≤ b + ε) : }
\NormalTok{    a ≤ b := }\KeywordTok{by}
  \KeywordTok{by\_contra}\NormalTok{ h2}
  \KeywordTok{push\_neg} \KeywordTok{at}\NormalTok{ h2}
  \KeywordTok{let}\NormalTok{ ε := (a {-} b) / 2}
  \KeywordTok{have}\NormalTok{ h3 : ε \textgreater{} 0 := }\KeywordTok{by}

    \SpecialCharTok{**}\WarningTok{done}\SpecialCharTok{::}

  \SpecialCharTok{**}\WarningTok{done}\SpecialCharTok{::}
\end{Highlighting}
\end{Shaded}

\end{inpt}

\begin{outpt}

\begin{Shaded}
\begin{Highlighting}[]
\InformationTok{R}\NormalTok{: Type u\_1}
\InformationTok{inst✝}\NormalTok{: Ring R}
\InformationTok{ab}\NormalTok{: ℝ}
\InformationTok{h1}\NormalTok{: ∀ (ε : ℝ), ε \textgreater{} 0 → }
\SpecialCharTok{\textgreater{}\textgreater{}}\NormalTok{  a ≤ b + ε}
\InformationTok{h2}\NormalTok{: b \textless{} a}
\InformationTok{ε}\NormalTok{: ℝ := (a {-} b) / 2}
\NormalTok{⊢ ε \textgreater{} 0}
\end{Highlighting}
\end{Shaded}

\end{outpt}

\hypertarget{step-4-1}{%
\subsubsection{Step 4}\label{step-4-1}}

Anyone reading a paragraph style proof such as ours would know that
dividing a number by two does not impact whether the resulting number if
positive or negative, but it still needs to be justified to lean. As
such, I use the \texttt{half\_pos} theorem with the \texttt{refine}
tactic to change the goal to what is currently shown in the infoview.
The \texttt{refine} tactic is useful because it tries to apply the
arguments it is given to the final goal and change the goal to whatever
is needed to meet the hypotheses in the arguments. In this case,
\texttt{half\_pos} claims that if you have some \(a > 0.\) then
\(\frac{a}{2} > 0.\) The \texttt{refine} tactic then applies the result
of that theorem and leaves us to show that \(a > 0,\) and lean is smart
enough to figure out that we actually need to show \(a - b > 0.\)

\begin{inpt}

\begin{Shaded}
\begin{Highlighting}[]
\KeywordTok{example}\NormalTok{ (a b : ℝ) (h1 : ∀ ε : ℝ, }
\NormalTok{    ε \textgreater{} 0 → a ≤ b + ε) : }
\NormalTok{    a ≤ b := }\KeywordTok{by}
  \KeywordTok{by\_contra}\NormalTok{ h2}
  \KeywordTok{push\_neg} \KeywordTok{at}\NormalTok{ h2}
  \KeywordTok{let}\NormalTok{ ε := (a {-} b) / 2}
  \KeywordTok{have}\NormalTok{ h3 : ε \textgreater{} 0 := }\KeywordTok{by}
\NormalTok{    refine half\_pos ?h}

      \SpecialCharTok{**}\WarningTok{done}\SpecialCharTok{::}

  \SpecialCharTok{**}\WarningTok{done}\SpecialCharTok{::}
\end{Highlighting}
\end{Shaded}

\end{inpt}

\begin{outpt}

\begin{Shaded}
\begin{Highlighting}[]
\InformationTok{R}\NormalTok{: Type u\_1}
\InformationTok{inst✝}\NormalTok{: Ring R}
\InformationTok{ab}\NormalTok{: ℝ}
\InformationTok{h1}\NormalTok{: ∀ (ε : ℝ), ε \textgreater{} 0 → }
\SpecialCharTok{\textgreater{}\textgreater{}}\NormalTok{  a ≤ b + ε}
\InformationTok{h2}\NormalTok{: b \textless{} a}
\InformationTok{ε}\NormalTok{: ℝ := (a {-} b) / 2}
\NormalTok{⊢ 0 \textless{} a {-} b}
\end{Highlighting}
\end{Shaded}

\end{outpt}

\hypertarget{step-5-1}{%
\subsubsection{Step 5}\label{step-5-1}}

The theorem I uses our second hypothesis to show that \(a - b > 0,\)
which finishes the proof for our third hypothesis and the goal switches
back to finding a contradiction.

\begin{inpt}

\begin{Shaded}
\begin{Highlighting}[]
\KeywordTok{example}\NormalTok{ (a b : ℝ) (h1 : ∀ ε : ℝ, }
\NormalTok{    ε \textgreater{} 0 → a ≤ b + ε) : }
\NormalTok{    a ≤ b := }\KeywordTok{by}
  \KeywordTok{by\_contra}\NormalTok{ h2}
  \KeywordTok{push\_neg} \KeywordTok{at}\NormalTok{ h2}
  \KeywordTok{let}\NormalTok{ ε := (a {-} b) / 2}
  \KeywordTok{have}\NormalTok{ h3 : ε \textgreater{} 0 := }\KeywordTok{by}
\NormalTok{    refine half\_pos ?h}
    \KeywordTok{exact}\NormalTok{ Iff.mpr sub\_pos h2}
    \KeywordTok{done}

  \SpecialCharTok{**}\WarningTok{done}\SpecialCharTok{::}
\end{Highlighting}
\end{Shaded}

\end{inpt}

\begin{outpt}

\begin{Shaded}
\begin{Highlighting}[]
\InformationTok{R}\NormalTok{: Type u\_1}
\InformationTok{inst✝}\NormalTok{: Ring R}
\InformationTok{ab}\NormalTok{: ℝ}
\InformationTok{h1}\NormalTok{: ∀ (ε : ℝ), ε \textgreater{} 0 → }
\SpecialCharTok{\textgreater{}\textgreater{}}\NormalTok{  a ≤ b + ε}
\InformationTok{h2}\NormalTok{: b \textless{} a}
\InformationTok{ε}\NormalTok{: ℝ := (a {-} b) / 2}
\InformationTok{h3}\NormalTok{: ε \textgreater{} 0}
\NormalTok{⊢ False}
\end{Highlighting}
\end{Shaded}

\end{outpt}

\hypertarget{step-6-1}{%
\subsubsection{Step 6}\label{step-6-1}}

I now try to lay out the fourth and final hypothesis which will be used
to find a contradiction with \texttt{h2}. This is another example of
something being quickly explained in the paragraph style proof, but
being more cumbersome to justify within lean.

\begin{inpt}

\begin{Shaded}
\begin{Highlighting}[]
\KeywordTok{example}\NormalTok{ (a b : ℝ) (h1 : ∀ ε : ℝ, }
\NormalTok{    ε \textgreater{} 0 → a ≤ b + ε) : }
\NormalTok{    a ≤ b := }\KeywordTok{by}
  \KeywordTok{by\_contra}\NormalTok{ h2}
  \KeywordTok{push\_neg} \KeywordTok{at}\NormalTok{ h2}
  \KeywordTok{let}\NormalTok{ ε := (a {-} b) / 2}
  \KeywordTok{have}\NormalTok{ h3 : ε \textgreater{} 0 := }\KeywordTok{by}
\NormalTok{    refine half\_pos ?h}
    \KeywordTok{exact}\NormalTok{ Iff.mpr sub\_pos h2}
    \KeywordTok{done}
  \KeywordTok{have}\NormalTok{ h4 : a ≤ b + ε := }\KeywordTok{by}

    \SpecialCharTok{**}\WarningTok{done}\SpecialCharTok{::}

  \SpecialCharTok{**}\WarningTok{done}\SpecialCharTok{::}
\end{Highlighting}
\end{Shaded}

\end{inpt}

\begin{outpt}

\begin{Shaded}
\begin{Highlighting}[]
\InformationTok{R}\NormalTok{: Type u\_1}
\InformationTok{inst✝}\NormalTok{: Ring R}
\InformationTok{ab}\NormalTok{: ℝ}
\InformationTok{h1}\NormalTok{: ∀ (ε : ℝ), ε \textgreater{} 0 → }
\SpecialCharTok{\textgreater{}\textgreater{}}\NormalTok{  a ≤ b + ε}
\InformationTok{h2}\NormalTok{: b \textless{} a}
\InformationTok{ε}\NormalTok{: ℝ := (a {-} b) / 2}
\InformationTok{h3}\NormalTok{: ε \textgreater{} 0}
\NormalTok{⊢ a ≤ b + ε}
\end{Highlighting}
\end{Shaded}

\end{outpt}

\hypertarget{step-7-1}{%
\subsubsection{Step 7}\label{step-7-1}}

I first apply \texttt{h1} which works has a similar effect as using the
refine tactic earler: it applies to result of an if then statement and
changes our goal to the if.

\begin{inpt}

\begin{Shaded}
\begin{Highlighting}[]
\KeywordTok{example}\NormalTok{ (a b : ℝ) (h1 : ∀ ε : ℝ, }
\NormalTok{    ε \textgreater{} 0 → a ≤ b + ε) : }
\NormalTok{    a ≤ b := }\KeywordTok{by}
  \KeywordTok{by\_contra}\NormalTok{ h2}
  \KeywordTok{push\_neg} \KeywordTok{at}\NormalTok{ h2}
  \KeywordTok{let}\NormalTok{ ε := (a {-} b) / 2}
  \KeywordTok{have}\NormalTok{ h3 : ε \textgreater{} 0 := }\KeywordTok{by}
\NormalTok{    refine half\_pos ?h}
    \KeywordTok{exact}\NormalTok{ Iff.mpr sub\_pos h2}
    \KeywordTok{done}
  \KeywordTok{have}\NormalTok{ h4 : a ≤ b + ε := }\KeywordTok{by}
    \KeywordTok{apply}\NormalTok{ h1}

    \SpecialCharTok{**}\WarningTok{done}\SpecialCharTok{::}

  \SpecialCharTok{**}\WarningTok{done}\SpecialCharTok{::}
\end{Highlighting}
\end{Shaded}

\end{inpt}

\begin{outpt}

\begin{Shaded}
\begin{Highlighting}[]
\InformationTok{R}\NormalTok{: Type u\_1}
\InformationTok{inst✝}\NormalTok{: Ring R}
\InformationTok{ab}\NormalTok{: ℝ}
\InformationTok{h1}\NormalTok{: ∀ (ε : ℝ), ε \textgreater{} 0 → }
\SpecialCharTok{\textgreater{}\textgreater{}}\NormalTok{  a ≤ b + ε}
\InformationTok{h2}\NormalTok{: b \textless{} a}
\InformationTok{ε}\NormalTok{: ℝ := (a {-} b) / 2}
\InformationTok{h3}\NormalTok{: ε \textgreater{} 0}
\NormalTok{⊢ ε \textgreater{} 0}
\end{Highlighting}
\end{Shaded}

\end{outpt}

\hypertarget{step-8}{%
\subsubsection{Step 8}\label{step-8}}

Now that our goal has been properly modified, \texttt{h3} is the only
other thing necessary to justify this hypothesis.

\begin{inpt}

\begin{Shaded}
\begin{Highlighting}[]
\KeywordTok{example}\NormalTok{ (a b : ℝ) (h1 : ∀ ε : ℝ, }
\NormalTok{    ε \textgreater{} 0 → a ≤ b + ε) : }
\NormalTok{    a ≤ b := }\KeywordTok{by}
  \KeywordTok{by\_contra}\NormalTok{ h2}
  \KeywordTok{push\_neg} \KeywordTok{at}\NormalTok{ h2}
  \KeywordTok{let}\NormalTok{ ε := (a {-} b) / 2}
  \KeywordTok{have}\NormalTok{ h3 : ε \textgreater{} 0 := }\KeywordTok{by}
\NormalTok{    refine half\_pos ?h}
    \KeywordTok{exact}\NormalTok{ Iff.mpr sub\_pos h2}
    \KeywordTok{done}
  \KeywordTok{have}\NormalTok{ h4 : a ≤ b + ε := }\KeywordTok{by}
    \KeywordTok{apply}\NormalTok{ h1}
    \KeywordTok{apply}\NormalTok{ h3}
    \KeywordTok{done}

  \SpecialCharTok{**}\WarningTok{done}\SpecialCharTok{::}
\end{Highlighting}
\end{Shaded}

\end{inpt}

\begin{outpt}

\begin{Shaded}
\begin{Highlighting}[]
\InformationTok{R}\NormalTok{: Type u\_1}
\InformationTok{inst✝}\NormalTok{: Ring R}
\InformationTok{ab}\NormalTok{: ℝ}
\InformationTok{h1}\NormalTok{: ∀ (ε : ℝ), ε \textgreater{} 0 → }
\SpecialCharTok{\textgreater{}\textgreater{}}\NormalTok{  a ≤ b + ε}
\InformationTok{h2}\NormalTok{: b \textless{} a}
\InformationTok{ε}\NormalTok{: ℝ := (a {-} b) / 2}
\InformationTok{h3}\NormalTok{: ε \textgreater{} 0}
\InformationTok{h4}\NormalTok{: a ≤ b + ε}
\NormalTok{⊢ False}
\end{Highlighting}
\end{Shaded}

\end{outpt}

\hypertarget{step-9}{%
\subsubsection{Step 9}\label{step-9}}

The \texttt{dsimp} tactic will do its best to automatically simplify
anything it is given, in this case it substitutes our specific epsilon
value in for the arbitrary epsilon. This will now allow us to use
\texttt{h4} to find our contradiction.

\begin{inpt}

\begin{Shaded}
\begin{Highlighting}[]
\KeywordTok{example}\NormalTok{ (a b : ℝ) (h1 : ∀ ε : ℝ, }
\NormalTok{    ε \textgreater{} 0 → a ≤ b + ε) : }
\NormalTok{    a ≤ b := }\KeywordTok{by}
  \KeywordTok{by\_contra}\NormalTok{ h2}
  \KeywordTok{push\_neg} \KeywordTok{at}\NormalTok{ h2}
  \KeywordTok{let}\NormalTok{ ε := (a {-} b) / 2}
  \KeywordTok{have}\NormalTok{ h3 : ε \textgreater{} 0 := }\KeywordTok{by}
\NormalTok{    refine half\_pos ?h}
    \KeywordTok{exact}\NormalTok{ Iff.mpr sub\_pos h2}
    \KeywordTok{done}
  \KeywordTok{have}\NormalTok{ h4 : a ≤ b + ε := }\KeywordTok{by}
    \KeywordTok{apply}\NormalTok{ h1}
    \KeywordTok{apply}\NormalTok{ h3}
    \KeywordTok{done}
\NormalTok{  dsimp }\KeywordTok{at}\NormalTok{ h4}

  \SpecialCharTok{**}\WarningTok{done}\SpecialCharTok{::}
\end{Highlighting}
\end{Shaded}

\end{inpt}

\begin{outpt}

\begin{Shaded}
\begin{Highlighting}[]
\InformationTok{R}\NormalTok{: Type u\_1}
\InformationTok{inst✝}\NormalTok{: Ring R}
\InformationTok{ab}\NormalTok{: ℝ}
\InformationTok{h1}\NormalTok{: ∀ (ε : ℝ), ε \textgreater{} 0 → }
\SpecialCharTok{\textgreater{}\textgreater{}}\NormalTok{  a ≤ b + ε}
\InformationTok{h2}\NormalTok{: b \textless{} a}
\InformationTok{ε}\NormalTok{: ℝ := (a {-} b) / 2}
\InformationTok{h3}\NormalTok{: ε \textgreater{} 0}
\InformationTok{h4}\NormalTok{: a ≤ b + (a {-} b) / 2}
\NormalTok{⊢ False}
\end{Highlighting}
\end{Shaded}

\end{outpt}

\hypertarget{step-10}{%
\subsubsection{Step 10}\label{step-10}}

The \texttt{linarith} tactic is quite powerful as it will attempt to
simplify the goal as well as hypotheses and then look for a
contradiction amongst the known hypotheses. This is one example where
lean actually requires quite a bit less explanation than a typical
proof. The majority of my paragraph style proof above was spent
simplifying and manipulating \texttt{h4} and \texttt{h2}, whereas in
lean I need to specify none of that! It is quite impressive that lean is
already able to do so much simplification and even find contradictions
with no user input. This ability will likely only increase in power in
the future, and some developments have even occured during the planning
and writing of this thesis that make other simplification tactics
substantially more powerful.

\begin{inpt}

\begin{Shaded}
\begin{Highlighting}[]
\KeywordTok{example}\NormalTok{ (a b : ℝ) (h1 : ∀ ε : ℝ, }
\NormalTok{    ε \textgreater{} 0 → a ≤ b + ε) : }
\NormalTok{    a ≤ b := }\KeywordTok{by}
  \KeywordTok{by\_contra}\NormalTok{ h2}
  \KeywordTok{push\_neg} \KeywordTok{at}\NormalTok{ h2}
  \KeywordTok{let}\NormalTok{ ε := (a {-} b) / 2}
  \KeywordTok{have}\NormalTok{ h3 : ε \textgreater{} 0 := }\KeywordTok{by}
\NormalTok{    refine half\_pos ?h}
    \KeywordTok{exact}\NormalTok{ Iff.mpr sub\_pos h2}
    \KeywordTok{done}
  \KeywordTok{have}\NormalTok{ h4 : a ≤ b + ε := }\KeywordTok{by}
    \KeywordTok{apply}\NormalTok{ h1}
    \KeywordTok{apply}\NormalTok{ h3}
    \KeywordTok{done}
\NormalTok{  dsimp }\KeywordTok{at}\NormalTok{ h4}
  \KeywordTok{linarith}
  \KeywordTok{done}
\end{Highlighting}
\end{Shaded}

\end{inpt}

\begin{outpt}

\begin{Shaded}
\begin{Highlighting}[]
\SpecialStringTok{No}\InformationTok{ }\SpecialStringTok{goals}
\end{Highlighting}
\end{Shaded}

\end{outpt}

\hypertarget{absolute-convergence}{%
\section{Absolute Convergence}\label{absolute-convergence}}

\hypertarget{paragraph-style-proof-2}{%
\subsection{Paragraph Style Proof}\label{paragraph-style-proof-2}}

\begin{thm}
Prove that \(\lim(x_{n}) = 0\) if and only if
\(\lim(\left|x_{n}\right|) = 0.\)

\end{thm}

\begin{proof}

\((\Longrightarrow)\) First assume that \(\lim(x_{n}) = 0.\) Then for
all \(\varepsilon > 0\) we know there exists a \(k_{n} \in \mathbb{N}\)
such that for all nautral numbers \(n > k_{n},\)
\(\left|x_{n} - 0\right| < \varepsilon.\) Thus
\(\left|x_{n}\right| < \varepsilon\) and also
\(\left|\left|x_{n}\right| - 0 \right| < \varepsilon,\) so
\(\lim(\left|x_{n}\right|) = 0.\)

\((\Longleftarrow)\) Now assume that \(\lim(\left|x_{n}\right|) = 0.\)
Then for all \(\varepsilon > 0\) we know there exists a
\(k_{n} \in \mathbb{N}\) such that for all nautral numbers
\(n > k_{n},\) \(\left|\left|x_{n}\right| - 0\right| < \varepsilon.\)
But
\(\left|\left|x_{n}\right| - 0\right| = \left|\left|x_{n}\right|\right| = \left|x_{n}\right| = \left|x_{n} - 0\right|.\)
So \(\left|x_{n} - 0\right| < \varepsilon\) and \(\lim(x_{n}) = 0.\)
\excl{~□}\qedhere

\end{proof}

\hypertarget{lean-proof-2}{%
\subsection{Lean Proof}\label{lean-proof-2}}

\hypertarget{setting-up-the-problem-1}{%
\subsubsection{Setting up the problem}\label{setting-up-the-problem-1}}

Lean does not include a built in epsilon definition of a limit for
sequences, so it is first necessary to define a limit in lean. I use the
following definition:

\begin{Shaded}
\begin{Highlighting}[]
\KeywordTok{def}\NormalTok{ ConvergesTo (s : ℕ → ℝ) (a : ℝ) :=}
\NormalTok{  ∀ ε \textgreater{} 0, ∃ N, ∀ n ≥ N, |s n {-} a| \textless{} ε}
\end{Highlighting}
\end{Shaded}

From this point we can set up our problem as normal.

\begin{inpt}

\begin{Shaded}
\begin{Highlighting}[]
\KeywordTok{example}\NormalTok{ (s1 : ℕ → ℝ) : }
\NormalTok{    ConvergesTo s1 (0 : ℝ) ↔}
\NormalTok{    ConvergesTo (abs s1) (0 : ℝ) }
\NormalTok{    := }\KeywordTok{by}

  \SpecialCharTok{**}\WarningTok{done}\SpecialCharTok{::}
\end{Highlighting}
\end{Shaded}

\end{inpt}

\begin{outpt}

\begin{Shaded}
\begin{Highlighting}[]
\InformationTok{s1}\NormalTok{: ℕ → ℝ}
\NormalTok{⊢ ConvergesTo s1 0 }
\SpecialCharTok{\textgreater{}\textgreater{}}\NormalTok{  ↔ ConvergesTo |s1| 0}
\end{Highlighting}
\end{Shaded}

\end{outpt}

\hypertarget{step-1-2}{%
\subsubsection{Step 1}\label{step-1-2}}

The first thing I ask lean to do is rewrite the definition of
convergence that I defined earler when it is used in our goal. This will
allow us to actually use and work towards the information in both
instances of \texttt{ConvergesTo} in the problem. The fully expanded
definition is shown in the infoview panel.

\begin{inpt}

\begin{Shaded}
\begin{Highlighting}[]
\KeywordTok{example}\NormalTok{ (s1 : ℕ → ℝ) : }
\NormalTok{    ConvergesTo s1 (0 : ℝ) ↔}
\NormalTok{    ConvergesTo (abs s1) (0 : ℝ) }
\NormalTok{    := }\KeywordTok{by}
  \KeywordTok{rw}\NormalTok{ [ConvergesTo]}
  \KeywordTok{rw}\NormalTok{ [ConvergesTo]}

  \SpecialCharTok{**}\WarningTok{done}\SpecialCharTok{::}
\end{Highlighting}
\end{Shaded}

\end{inpt}

\begin{outpt}

\begin{Shaded}
\begin{Highlighting}[]
\InformationTok{s1}\NormalTok{: ℕ → ℝ}
\NormalTok{⊢ (∀ (ε : ℝ), ε \textgreater{} 0 → }
\SpecialCharTok{\textgreater{}\textgreater{}}\NormalTok{    ∃ N, ∀ (n : ℕ), }
\SpecialCharTok{\textgreater{}\textgreater{}}\NormalTok{    n ≥ N → }
\SpecialCharTok{\textgreater{}\textgreater{}}\NormalTok{    |s1 n {-} 0| \textless{} ε) ↔}
\SpecialCharTok{\textgreater{}\textgreater{}}\NormalTok{    ∀ (ε : ℝ), ε \textgreater{} 0 → }
\SpecialCharTok{\textgreater{}\textgreater{}}\NormalTok{    ∃ N, ∀ (n : ℕ), }
\SpecialCharTok{\textgreater{}\textgreater{}}\NormalTok{    n ≥ N → }
\SpecialCharTok{\textgreater{}\textgreater{}}\NormalTok{    |abs s1 n {-} 0| \textless{} ε}
\end{Highlighting}
\end{Shaded}

\end{outpt}

\hypertarget{step-2-2}{%
\subsubsection{Step 2}\label{step-2-2}}

Here I set up a hypothesis which will later be used to modify both sides
of the if and only if statement into something that is equal to the
other.

\begin{inpt}

\begin{Shaded}
\begin{Highlighting}[]
\KeywordTok{example}\NormalTok{ (s1 : ℕ → ℝ) : }
\NormalTok{    ConvergesTo s1 (0 : ℝ) ↔}
\NormalTok{    ConvergesTo (abs s1) (0 : ℝ) }
\NormalTok{    := }\KeywordTok{by}
  \KeywordTok{rw}\NormalTok{ [ConvergesTo]}
  \KeywordTok{rw}\NormalTok{ [ConvergesTo]}
  \KeywordTok{have}\NormalTok{ h3 (x : ℕ) : |s1 x| = }
\NormalTok{      |abs s1 x| := }\KeywordTok{by}
    
    \KeywordTok{done}

  \SpecialCharTok{**}\WarningTok{done}\SpecialCharTok{::}
\end{Highlighting}
\end{Shaded}

\end{inpt}

\begin{outpt}

\begin{Shaded}
\begin{Highlighting}[]
\InformationTok{s1}\NormalTok{: ℕ → ℝ}
\InformationTok{x}\NormalTok{: ℕ}
\NormalTok{⊢ |s1 x| = |abs s1 x|}
\end{Highlighting}
\end{Shaded}

\end{outpt}

\hypertarget{step-3-2}{%
\subsubsection{Step 3}\label{step-3-2}}

In this instance lean essentially already knows that our goal is true,
and only need to be told to simplify using the definition of absolute
value in order to verify this. While it is impressive that lean requires
little guidance, seeing some of the other things lean is capable of
leaves me a bit underwhelmed that lean requires any input here. Because
lean is still being developed there may come a time where simple
statements like this are automatically verified without any user input.

\begin{inpt}

\begin{Shaded}
\begin{Highlighting}[]
\KeywordTok{example}\NormalTok{ (s1 : ℕ → ℝ) : }
\NormalTok{    ConvergesTo s1 (0 : ℝ) ↔}
\NormalTok{    ConvergesTo (abs s1) (0 : ℝ) }
\NormalTok{    := }\KeywordTok{by}
  \KeywordTok{rw}\NormalTok{ [ConvergesTo]}
  \KeywordTok{rw}\NormalTok{ [ConvergesTo]}
  \KeywordTok{have}\NormalTok{ h3 (x : ℕ) : |s1 x| = }
\NormalTok{      |abs s1 x| := }\KeywordTok{by}
    \KeywordTok{simp}\NormalTok{ [abs]}
    \KeywordTok{done}

  \SpecialCharTok{**}\WarningTok{done}\SpecialCharTok{::}
\end{Highlighting}
\end{Shaded}

\end{inpt}

\begin{outpt}

\begin{Shaded}
\begin{Highlighting}[]
\InformationTok{s1}\NormalTok{: ℕ → ℝ}
\InformationTok{h3}\NormalTok{: ∀ (x : ℕ), |s1 x| = }
\SpecialCharTok{\textgreater{}\textgreater{}}\NormalTok{  |abs s1 x|}
\NormalTok{⊢ (∀ (ε : ℝ), ε \textgreater{} 0 → }
\SpecialCharTok{\textgreater{}\textgreater{}}\NormalTok{    ∃ N, ∀ (n : ℕ), }
\SpecialCharTok{\textgreater{}\textgreater{}}\NormalTok{    n ≥ N → }
\SpecialCharTok{\textgreater{}\textgreater{}}\NormalTok{    |s1 n {-} 0| \textless{} ε) ↔}
\SpecialCharTok{\textgreater{}\textgreater{}}\NormalTok{    ∀ (ε : ℝ), ε \textgreater{} 0 → }
\SpecialCharTok{\textgreater{}\textgreater{}}\NormalTok{    ∃ N, ∀ (n : ℕ), }
\SpecialCharTok{\textgreater{}\textgreater{}}\NormalTok{    n ≥ N → }
\SpecialCharTok{\textgreater{}\textgreater{}}\NormalTok{    |abs s1 n {-} 0| \textless{} ε}
\end{Highlighting}
\end{Shaded}

\end{outpt}

\hypertarget{step-4-2}{%
\subsubsection{Step 4}\label{step-4-2}}

Here the \texttt{Iff.intro} tactic splits up the if and only if
statement in the goal and allows us to prove each direction
individually, as is often done in a paragraph style proof.

\begin{inpt}

\begin{Shaded}
\begin{Highlighting}[]
\KeywordTok{example}\NormalTok{ (s1 : ℕ → ℝ) : }
\NormalTok{    ConvergesTo s1 (0 : ℝ) ↔}
\NormalTok{    ConvergesTo (abs s1) (0 : ℝ) }
\NormalTok{    := }\KeywordTok{by}
  \KeywordTok{rw}\NormalTok{ [ConvergesTo]}
  \KeywordTok{rw}\NormalTok{ [ConvergesTo]}
  \KeywordTok{have}\NormalTok{ h3 (x : ℕ) : |s1 x| = }
\NormalTok{      |abs s1 x| := }\KeywordTok{by}
    \KeywordTok{simp}\NormalTok{ [abs]}
    \KeywordTok{done}
  \KeywordTok{apply}\NormalTok{ Iff.intro}
\NormalTok{  · }\CommentTok{{-}{-}Forwards}

\NormalTok{  · }\CommentTok{{-}{-}Reverse}

  \SpecialCharTok{**}\WarningTok{done}\SpecialCharTok{::}
\end{Highlighting}
\end{Shaded}

\end{inpt}

\begin{outpt}

\begin{Shaded}
\begin{Highlighting}[]
\InformationTok{s1}\NormalTok{: ℕ → ℝ}
\InformationTok{h3}\NormalTok{: ∀ (x : ℕ), |s1 x| = }
\SpecialCharTok{\textgreater{}\textgreater{}}\NormalTok{  |abs s1 x|}
\NormalTok{⊢ (∀ (ε : ℝ), ε \textgreater{} 0 → }
\SpecialCharTok{\textgreater{}\textgreater{}}\NormalTok{    ∃ N, ∀ (n : ℕ), }
\SpecialCharTok{\textgreater{}\textgreater{}}\NormalTok{    n ≥ N → }
\SpecialCharTok{\textgreater{}\textgreater{}}\NormalTok{    |s1 n {-} 0| \textless{} ε) →}
\SpecialCharTok{\textgreater{}\textgreater{}}\NormalTok{    ∀ (ε : ℝ), ε \textgreater{} 0 → }
\SpecialCharTok{\textgreater{}\textgreater{}}\NormalTok{    ∃ N, ∀ (n : ℕ), }
\SpecialCharTok{\textgreater{}\textgreater{}}\NormalTok{    n ≥ N → }
\SpecialCharTok{\textgreater{}\textgreater{}}\NormalTok{    |abs s1 n {-} 0| \textless{} ε}
\end{Highlighting}
\end{Shaded}

\end{outpt}

\hypertarget{step-5-2}{%
\subsubsection{Step 5}\label{step-5-2}}

The \texttt{intro} tactic applied here allows me to assume the if part
of an if then statement and automatically names it with the hypothesis
name I give it.

\begin{inpt}

\begin{Shaded}
\begin{Highlighting}[]
\KeywordTok{example}\NormalTok{ (s1 : ℕ → ℝ) : }
\NormalTok{    ConvergesTo s1 (0 : ℝ) ↔}
\NormalTok{    ConvergesTo (abs s1) (0 : ℝ) }
\NormalTok{    := }\KeywordTok{by}
  \KeywordTok{rw}\NormalTok{ [ConvergesTo]}
  \KeywordTok{rw}\NormalTok{ [ConvergesTo]}
  \KeywordTok{have}\NormalTok{ h3 (x : ℕ) : |s1 x| = }
\NormalTok{      |abs s1 x| := }\KeywordTok{by}
    \KeywordTok{simp}\NormalTok{ [abs]}
    \KeywordTok{done}
  \KeywordTok{apply}\NormalTok{ Iff.intro}
\NormalTok{  · }\CommentTok{{-}{-}Forwards}
\NormalTok{    intro h1}

\NormalTok{  · }\CommentTok{{-}{-}Reverse}

  \SpecialCharTok{**}\WarningTok{done}\SpecialCharTok{::}
\end{Highlighting}
\end{Shaded}

\end{inpt}

\begin{outpt}

\begin{Shaded}
\begin{Highlighting}[]
\InformationTok{s1}\NormalTok{: ℕ → ℝ}
\InformationTok{h3}\NormalTok{: ∀ (x : ℕ), |s1 x| = }
\SpecialCharTok{\textgreater{}\textgreater{}}\NormalTok{  |abs s1 x|}
\InformationTok{h1}\NormalTok{: ∀ (ε : ℝ), ε \textgreater{} 0 → }
\SpecialCharTok{\textgreater{}\textgreater{}}\NormalTok{  ∃ N, ∀ (n : ℕ), }
\SpecialCharTok{\textgreater{}\textgreater{}}\NormalTok{  n ≥ N → |s1 n {-} 0| \textless{} ε}
\NormalTok{⊢ ∀ (ε : ℝ), ε \textgreater{} 0 → }
\SpecialCharTok{\textgreater{}\textgreater{}}\NormalTok{    ∃ N, ∀ (n : ℕ), }
\SpecialCharTok{\textgreater{}\textgreater{}}\NormalTok{    n ≥ N → }
\SpecialCharTok{\textgreater{}\textgreater{}}\NormalTok{    |abs s1 n {-} 0| \textless{} ε}
\end{Highlighting}
\end{Shaded}

\end{outpt}

\hypertarget{step-6-2}{%
\subsubsection{Step 6}\label{step-6-2}}

With the \texttt{simp} tactic, lean attempts to simplify the current
goal. In this case, the \(\left|\left|s1_{n}\right| - 0\right|\) is
simplified to \(\left|\left|s1_{n}\right|\right|.\) This is now where
our \texttt{h3} hypothesis can be applied, but I will first attempt to
simplify \texttt{h1}.

\begin{inpt}

\begin{Shaded}
\begin{Highlighting}[]
\KeywordTok{example}\NormalTok{ (s1 : ℕ → ℝ) : }
\NormalTok{    ConvergesTo s1 (0 : ℝ) ↔}
\NormalTok{    ConvergesTo (abs s1) (0 : ℝ) }
\NormalTok{    := }\KeywordTok{by}
  \KeywordTok{rw}\NormalTok{ [ConvergesTo]}
  \KeywordTok{rw}\NormalTok{ [ConvergesTo]}
  \KeywordTok{have}\NormalTok{ h3 (x : ℕ) : |s1 x| = }
\NormalTok{      |abs s1 x| := }\KeywordTok{by}
    \KeywordTok{simp}\NormalTok{ [abs]}
    \KeywordTok{done}
  \KeywordTok{apply}\NormalTok{ Iff.intro}
\NormalTok{  · }\CommentTok{{-}{-}Forwards}
\NormalTok{    intro h1}
    \KeywordTok{simp}

\NormalTok{  · }\CommentTok{{-}{-}Reverse}

  \SpecialCharTok{**}\WarningTok{done}\SpecialCharTok{::}
\end{Highlighting}
\end{Shaded}

\end{inpt}

\begin{outpt}

\begin{Shaded}
\begin{Highlighting}[]
\InformationTok{s1}\NormalTok{: ℕ → ℝ}
\InformationTok{h3}\NormalTok{: ∀ (x : ℕ), |s1 x| = }
\SpecialCharTok{\textgreater{}\textgreater{}}\NormalTok{  |abs s1 x|}
\InformationTok{h1}\NormalTok{: ∀ (ε : ℝ), ε \textgreater{} 0 → }
\SpecialCharTok{\textgreater{}\textgreater{}}\NormalTok{  ∃ N, ∀ (n : ℕ), }
\SpecialCharTok{\textgreater{}\textgreater{}}\NormalTok{  n ≥ N → |s1 n {-} 0| \textless{} ε}
\NormalTok{⊢ ∀ (ε : ℝ), 0 \textless{} ε → }
\SpecialCharTok{\textgreater{}\textgreater{}}\NormalTok{    ∃ N, ∀ (n : ℕ), }
\SpecialCharTok{\textgreater{}\textgreater{}}\NormalTok{    N ≤ n → }
\SpecialCharTok{\textgreater{}\textgreater{}}\NormalTok{    |abs s1 n| \textless{} ε}
\end{Highlighting}
\end{Shaded}

\end{outpt}

\hypertarget{step-7-2}{%
\subsubsection{Step 7}\label{step-7-2}}

The \texttt{simp} tactic has the same effect as in the previous step,
but this time it is working on \texttt{h1} rather than the end goal. By
default \texttt{simp} will attempt to work on the goal but if asked to
it will attempt to simplify hypotheses as well.

\begin{inpt}

\begin{Shaded}
\begin{Highlighting}[]
\KeywordTok{example}\NormalTok{ (s1 : ℕ → ℝ) : }
\NormalTok{    ConvergesTo s1 (0 : ℝ) ↔}
\NormalTok{    ConvergesTo (abs s1) (0 : ℝ) }
\NormalTok{    := }\KeywordTok{by}
  \KeywordTok{rw}\NormalTok{ [ConvergesTo]}
  \KeywordTok{rw}\NormalTok{ [ConvergesTo]}
  \KeywordTok{have}\NormalTok{ h3 (x : ℕ) : |s1 x| = }
\NormalTok{      |abs s1 x| := }\KeywordTok{by}
    \KeywordTok{simp}\NormalTok{ [abs]}
    \KeywordTok{done}
  \KeywordTok{apply}\NormalTok{ Iff.intro}
\NormalTok{  · }\CommentTok{{-}{-}Forwards}
\NormalTok{    intro h1}
    \KeywordTok{simp}
    \KeywordTok{simp} \KeywordTok{at}\NormalTok{ h1}

\NormalTok{  · }\CommentTok{{-}{-}Reverse}

  \SpecialCharTok{**}\WarningTok{done}\SpecialCharTok{::}
\end{Highlighting}
\end{Shaded}

\end{inpt}

\begin{outpt}

\begin{Shaded}
\begin{Highlighting}[]
\InformationTok{s1}\NormalTok{: ℕ → ℝ}
\InformationTok{h3}\NormalTok{: ∀ (x : ℕ), |s1 x| = }
\SpecialCharTok{\textgreater{}\textgreater{}}\NormalTok{  |abs s1 x|}
\InformationTok{h1}\NormalTok{: ∀ (ε : ℝ), 0 \textless{} ε → }
\SpecialCharTok{\textgreater{}\textgreater{}}\NormalTok{  ∃ N, ∀ (n : ℕ), }
\SpecialCharTok{\textgreater{}\textgreater{}}\NormalTok{  N ≤ n → |s1 n| \textless{} ε}
\NormalTok{⊢ ∀ (ε : ℝ), 0 \textless{} ε → }
\SpecialCharTok{\textgreater{}\textgreater{}}\NormalTok{    ∃ N, ∀ (n : ℕ), }
\SpecialCharTok{\textgreater{}\textgreater{}}\NormalTok{    N ≤ n → }
\SpecialCharTok{\textgreater{}\textgreater{}}\NormalTok{    |abs s1 n| \textless{} ε}
\end{Highlighting}
\end{Shaded}

\end{outpt}

\hypertarget{step-8-1}{%
\subsubsection{Step 8}\label{step-8-1}}

We can now use the reverse direction of \texttt{h3} to simplify our goal
further. Notice that the leftwards facing arrow is necessary, as lean
typically tries to apply equalities from left to right. This means if
the left side of the equality does not match what lean is attempting to
replace, lean will not be able to rewrite in other terms.

\begin{inpt}

\begin{Shaded}
\begin{Highlighting}[]
\KeywordTok{example}\NormalTok{ (s1 : ℕ → ℝ) : }
\NormalTok{    ConvergesTo s1 (0 : ℝ) ↔}
\NormalTok{    ConvergesTo (abs s1) (0 : ℝ) }
\NormalTok{    := }\KeywordTok{by}
  \KeywordTok{rw}\NormalTok{ [ConvergesTo]}
  \KeywordTok{rw}\NormalTok{ [ConvergesTo]}
  \KeywordTok{have}\NormalTok{ h3 (x : ℕ) : |s1 x| = }
\NormalTok{      |abs s1 x| := }\KeywordTok{by}
    \KeywordTok{simp}\NormalTok{ [abs]}
    \KeywordTok{done}
  \KeywordTok{apply}\NormalTok{ Iff.intro}
\NormalTok{  · }\CommentTok{{-}{-}Forwards}
\NormalTok{    intro h1}
    \KeywordTok{simp}
    \KeywordTok{simp} \KeywordTok{at}\NormalTok{ h1}
    \KeywordTok{simp}\NormalTok{ [← h3]}

\NormalTok{  · }\CommentTok{{-}{-}Reverse}

  \SpecialCharTok{**}\WarningTok{done}\SpecialCharTok{::}
\end{Highlighting}
\end{Shaded}

\end{inpt}

\begin{outpt}

\begin{Shaded}
\begin{Highlighting}[]
\InformationTok{s1}\NormalTok{: ℕ → ℝ}
\InformationTok{h3}\NormalTok{: ∀ (x : ℕ), |s1 x| = }
\SpecialCharTok{\textgreater{}\textgreater{}}\NormalTok{  |abs s1 x|}
\InformationTok{h1}\NormalTok{: ∀ (ε : ℝ), 0 \textless{} ε → }
\SpecialCharTok{\textgreater{}\textgreater{}}\NormalTok{  ∃ N, ∀ (n : ℕ), }
\SpecialCharTok{\textgreater{}\textgreater{}}\NormalTok{  N ≤ n → |s1 n| \textless{} ε}
\NormalTok{⊢ ∀ (ε : ℝ), 0 \textless{} ε → }
\SpecialCharTok{\textgreater{}\textgreater{}}\NormalTok{    ∃ N, ∀ (n : ℕ), }
\SpecialCharTok{\textgreater{}\textgreater{}}\NormalTok{    N ≤ n → }
\SpecialCharTok{\textgreater{}\textgreater{}}\NormalTok{    |s1 n| \textless{} ε}
\end{Highlighting}
\end{Shaded}

\end{outpt}

\hypertarget{step-9-1}{%
\subsubsection{Step 9}\label{step-9-1}}

The simplification done over the last few steps has modified both
\texttt{h1} and our goal to be the same thing. Since we are assuming
\texttt{h1} to be true, this allows us to apply that hypothesis and
complete the first direction of out goal. Upon completion of the first
goal, lean automatically begins displaying the second goal, which can be
solved quite similarly to the first.

\begin{inpt}

\begin{Shaded}
\begin{Highlighting}[]
\KeywordTok{example}\NormalTok{ (s1 : ℕ → ℝ) : }
\NormalTok{    ConvergesTo s1 (0 : ℝ) ↔}
\NormalTok{    ConvergesTo (abs s1) (0 : ℝ) }
\NormalTok{    := }\KeywordTok{by}
  \KeywordTok{rw}\NormalTok{ [ConvergesTo]}
  \KeywordTok{rw}\NormalTok{ [ConvergesTo]}
  \KeywordTok{have}\NormalTok{ h3 (x : ℕ) : |s1 x| = }
\NormalTok{      |abs s1 x| := }\KeywordTok{by}
    \KeywordTok{simp}\NormalTok{ [abs]}
    \KeywordTok{done}
  \KeywordTok{apply}\NormalTok{ Iff.intro}
\NormalTok{  · }\CommentTok{{-}{-}Forwards}
\NormalTok{    intro h1}
    \KeywordTok{simp}
    \KeywordTok{simp} \KeywordTok{at}\NormalTok{ h1}
    \KeywordTok{simp}\NormalTok{ [← h3]}
    \KeywordTok{apply}\NormalTok{ h1}
\NormalTok{  · }\CommentTok{{-}{-}Reverse}

  \SpecialCharTok{**}\WarningTok{done}\SpecialCharTok{::}
\end{Highlighting}
\end{Shaded}

\end{inpt}

\begin{outpt}

\begin{Shaded}
\begin{Highlighting}[]
\InformationTok{s1}\NormalTok{: ℕ → ℝ}
\InformationTok{h3}\NormalTok{: ∀ (x : ℕ), |s1 x| = }
\SpecialCharTok{\textgreater{}\textgreater{}}\NormalTok{  |abs s1 x|}
\NormalTok{⊢ (∀ (ε : ℝ), ε \textgreater{} 0 → }
\SpecialCharTok{\textgreater{}\textgreater{}}\NormalTok{    ∃ N, ∀ (n : ℕ), }
\SpecialCharTok{\textgreater{}\textgreater{}}\NormalTok{    n ≥ N → }
\SpecialCharTok{\textgreater{}\textgreater{}}\NormalTok{    |abs s1 n {-} 0| \textless{} ε) →}
\SpecialCharTok{\textgreater{}\textgreater{}}\NormalTok{    ∀ (ε : ℝ), ε \textgreater{} 0 → }
\SpecialCharTok{\textgreater{}\textgreater{}}\NormalTok{    ∃ N, ∀ (n : ℕ), }
\SpecialCharTok{\textgreater{}\textgreater{}}\NormalTok{    n ≥ N → }
\SpecialCharTok{\textgreater{}\textgreater{}}\NormalTok{    |s1 n {-} 0| \textless{} ε}
\end{Highlighting}
\end{Shaded}

\end{outpt}

\hypertarget{step-10-1}{%
\subsubsection{Step 10}\label{step-10-1}}

With this direction I try to simplify in the same ways as before, but
instead of using the leftwards direction of the euality in \texttt{h3},
I use the rightwards direction. This means that arrow does not need to
be included and once again he have \texttt{h1} equal to our current
goal.

\begin{inpt}

\begin{Shaded}
\begin{Highlighting}[]
\KeywordTok{example}\NormalTok{ (s1 : ℕ → ℝ) : }
\NormalTok{    ConvergesTo s1 (0 : ℝ) ↔}
\NormalTok{    ConvergesTo (abs s1) (0 : ℝ) }
\NormalTok{    := }\KeywordTok{by}
  \KeywordTok{rw}\NormalTok{ [ConvergesTo]}
  \KeywordTok{rw}\NormalTok{ [ConvergesTo]}
  \KeywordTok{have}\NormalTok{ h3 (x : ℕ) : |s1 x| = }
\NormalTok{      |abs s1 x| := }\KeywordTok{by}
    \KeywordTok{simp}\NormalTok{ [abs]}
    \KeywordTok{done}
  \KeywordTok{apply}\NormalTok{ Iff.intro}
\NormalTok{  · }\CommentTok{{-}{-}Forwards}
\NormalTok{    intro h1}
    \KeywordTok{simp}
    \KeywordTok{simp} \KeywordTok{at}\NormalTok{ h1}
    \KeywordTok{simp}\NormalTok{ [← h3]}
    \KeywordTok{apply}\NormalTok{ h1}
\NormalTok{  · }\CommentTok{{-}{-}Reverse}
\NormalTok{    intro h1}
    \KeywordTok{simp}
    \KeywordTok{simp} \KeywordTok{at}\NormalTok{ h1}
\NormalTok{    simp\_rw [h3]}

  \SpecialCharTok{**}\WarningTok{done}\SpecialCharTok{::}
\end{Highlighting}
\end{Shaded}

\end{inpt}

\begin{outpt}

\begin{Shaded}
\begin{Highlighting}[]
\InformationTok{s1}\NormalTok{: ℕ → ℝ}
\InformationTok{h3}\NormalTok{: ∀ (x : ℕ), |s1 x| = }
\SpecialCharTok{\textgreater{}\textgreater{}}\NormalTok{  |abs s1 x|}
\InformationTok{h1}\NormalTok{: ∀ (ε : ℝ), 0 \textless{} ε → }
\SpecialCharTok{\textgreater{}\textgreater{}}\NormalTok{  ∃ N, ∀ (n : ℕ), }
\SpecialCharTok{\textgreater{}\textgreater{}}\NormalTok{  N ≤ n → |abs s1 n| \textless{} ε}
\NormalTok{⊢ ∀ (ε : ℝ), 0 \textless{} ε → }
\SpecialCharTok{\textgreater{}\textgreater{}}\NormalTok{    ∃ N, ∀ (n : ℕ),}
\SpecialCharTok{\textgreater{}\textgreater{}}\NormalTok{    N ≤ n → }
\SpecialCharTok{\textgreater{}\textgreater{}}\NormalTok{    |abs s1 n| \textless{} ε}
\end{Highlighting}
\end{Shaded}

\end{outpt}

\hypertarget{step-11}{%
\subsubsection{Step 11}\label{step-11}}

With a hypothesis equal to our goal, we are able to apply the hypothesis
and prove the other direction of the if and only if statement,
completing the proof.

\begin{inpt}

\begin{Shaded}
\begin{Highlighting}[]
\KeywordTok{example}\NormalTok{ (s1 : ℕ → ℝ) : }
\NormalTok{    ConvergesTo s1 (0 : ℝ) ↔}
\NormalTok{    ConvergesTo (abs s1) (0 : ℝ) }
\NormalTok{    := }\KeywordTok{by}
  \KeywordTok{rw}\NormalTok{ [ConvergesTo]}
  \KeywordTok{rw}\NormalTok{ [ConvergesTo]}
  \KeywordTok{have}\NormalTok{ h3 (x : ℕ) : |s1 x| = }
\NormalTok{      |abs s1 x| := }\KeywordTok{by}
    \KeywordTok{simp}\NormalTok{ [abs]}
    \KeywordTok{done}
  \KeywordTok{apply}\NormalTok{ Iff.intro}
\NormalTok{  · }\CommentTok{{-}{-}Forwards}
\NormalTok{    intro h1}
    \KeywordTok{simp}
    \KeywordTok{simp} \KeywordTok{at}\NormalTok{ h1}
    \KeywordTok{simp}\NormalTok{ [← h3]}
    \KeywordTok{apply}\NormalTok{ h1}
\NormalTok{  · }\CommentTok{{-}{-}Reverse}
\NormalTok{    intro h1}
    \KeywordTok{simp}
    \KeywordTok{simp} \KeywordTok{at}\NormalTok{ h1}
\NormalTok{    simp\_rw [h3]}
    \KeywordTok{apply}\NormalTok{ h1}
  \KeywordTok{done}
\end{Highlighting}
\end{Shaded}

\end{inpt}

\begin{outpt}

\begin{Shaded}
\begin{Highlighting}[]
\SpecialStringTok{No}\InformationTok{ }\SpecialStringTok{goals}
\end{Highlighting}
\end{Shaded}

\end{outpt}

\hypertarget{convergence-of-a-specific-sequence}{%
\section{Convergence of a Specific
Sequence}\label{convergence-of-a-specific-sequence}}

The following is an example of one situation where lean is somewhat
lacking in comparison to a paragraph style proof. The paragraph style
proof is able to quickly and easily prove the desired end goal, but lean
has to work around a lot of the simple rewriting we would do in a normal
proof. In this attempt to prove the convergence of a specific sequence,
there were many issues with simplification involving arbitrary variables
and the change from natural numbers to real numbers. These sorts of
things can be easily explained in a paragraph style proof, but required
significant work to prove in lean.

I mentioned earlier that lean's ability to simplify and make connections
without user input was advancing quickly, and I encountered this when
working on this problem. I originally had great difficulty getting lean
to accept that \(2 = \frac{2(n + 1)}{n + 1},\) which is something which
can easily be explained in a typical proof, but lean has recently
strengthened a tactic that renders much of my work here unnecessary. The
\texttt{field\_simp} tactic tries to simplify the current goal using
what is known about all fields, and since we are working with the real
numbers we are able to take advantage of this. I was not able to use
this tactic since it was changed while I was working on the project, but
seeing how quickly lean is progressing is very promising.

Lean internally defines limits using filters and topology rather than
the real analysis approach of epsilons, so the approach I was taking
here is not the optimal approach for theorems involving limits in lean.
While this high level definition of a limit is very useful for the
people who know how to use it, it makes lean more difficult to use for
those who have not yet studied topology. Definitions such as this start
to portray that lean is not really something meant to be used for lower
level mathematics, but rather complex and high level proofs.

\hypertarget{paragraph-style-proof-3}{%
\subsection{Paragraph Style Proof}\label{paragraph-style-proof-3}}

\begin{thm}
Prove that \(\lim(\frac{2n}{n + 1}) = 2.\)

\end{thm}

\begin{proof}

Let \(\varepsilon > 0\) and choose \(k > \frac{1}{\varepsilon} - 1\)
where \(k \in \mathbb{N}\) by the Archimedean Property. Then for
\(n > k:\) \begin{align*}
\left|\frac{2n}{n + 1} - 2\right| & = \left|\frac{2n}{n + 1} - \frac{2(n + 1)}{n + 1}\right|\\
& = \left|\frac{-1}{n + 1}\right|\\
& = \frac{1}{n + 1}\\
& < \frac{1}{k + 1}\\
& < \frac{1}{\frac{1}{\varepsilon} - 1 + 1} = \varepsilon.
\end{align*} Thus we have that \(\lim(\frac{2n}{n + 1}) = 2.\)
\excl{~□}\qedhere

\end{proof}

\hypertarget{lean-proof-3}{%
\subsection{Lean Proof}\label{lean-proof-3}}

\begin{Shaded}
\begin{Highlighting}[]
\KeywordTok{example}\NormalTok{ : ConvergesTo (fun (n : ℕ) ↦ }
\NormalTok{    ((2 * n) / (n + 1))) 2 := }\KeywordTok{by}
\NormalTok{  intro ε}
\NormalTok{  intro h1}
  \KeywordTok{obtain}\NormalTok{ ⟨k, h13⟩ := }
\NormalTok{    exists\_nat\_gt (2 / ε {-} 1) }\CommentTok{{-}{-}Archimedean Property}
\NormalTok{  use k}
\NormalTok{  intro n}
\NormalTok{  intro h2}
\NormalTok{  dsimp}
  \KeywordTok{have}\NormalTok{ h3 : (2 : ℝ) = 2 * ((n + 1) / (n + 1)) := }\KeywordTok{by}
    \KeywordTok{have}\NormalTok{ h4 : ((n + 1) / (n + 1)) = }
\NormalTok{        (n + 1) * ((n + 1) : ℝ)⁻¹ := }\KeywordTok{by}
      \KeywordTok{rfl}
      \KeywordTok{done}
    \KeywordTok{rw}\NormalTok{ [h4]}
    \KeywordTok{have}\NormalTok{ h5 : (n + 1) * ((n + 1) : ℝ)⁻¹ = 1 := }\KeywordTok{by}
      \KeywordTok{rw}\NormalTok{ [mul\_inv\_cancel]}
      \KeywordTok{exact}\NormalTok{ Nat.cast\_add\_one\_ne\_zero n}
      \KeywordTok{done}
    \KeywordTok{rw}\NormalTok{ [h5]}
    \KeywordTok{exact}\NormalTok{ Eq.symm (mul\_one 2)}
    \KeywordTok{done}
\NormalTok{  nth\_rewrite 2 [h3]}
  \KeywordTok{have}\NormalTok{ h6 : 2 * ((↑n + 1) : ℝ) / (↑n + 1) = }
\NormalTok{      ((2 * n) + 2) / (n + 1) := }\KeywordTok{by}
    \KeywordTok{rw}\NormalTok{ [Distribute n]}
    \KeywordTok{done}
  \KeywordTok{have}\NormalTok{ h7 : 2 * (((↑n + 1) : ℝ) / (↑n + 1)) = }
\NormalTok{      2 * (↑n + 1) / (↑n + 1) := }\KeywordTok{by}
    \KeywordTok{rw}\NormalTok{ [← mul\_div\_assoc 2 ((n + 1) : ℝ) ((n + 1) : ℝ)]}
    \KeywordTok{done}
  \KeywordTok{rw}\NormalTok{ [h7]}
  \KeywordTok{rw}\NormalTok{ [h6]}
  \KeywordTok{rw}\NormalTok{ [div\_sub\_div\_same (2 * n : ℝ) (2 * n + 2) (n + 1)]}
  \KeywordTok{rw}\NormalTok{ [sub\_add\_cancel\textquotesingle{}]}
  \KeywordTok{rw}\NormalTok{ [abs\_div]}
  \KeywordTok{simp}
  \KeywordTok{have}\NormalTok{ h8 : |(↑n + 1 : ℝ)| = ↑n + 1 := }\KeywordTok{by}
    \KeywordTok{simp}
    \KeywordTok{apply}\NormalTok{ LT.lt.le (Nat.cast\_add\_one\_pos ↑n)}
    \KeywordTok{done}
  \KeywordTok{rw}\NormalTok{ [h8]}
  \KeywordTok{have}\NormalTok{ h9 : (2 : ℝ) / (↑n + 1) ≤ 2 / (k + 1) := }\KeywordTok{by}
    \KeywordTok{apply}\NormalTok{ div\_le\_div\_of\_le\_left}
\NormalTok{    · }\CommentTok{{-}{-}case 1}
      \KeywordTok{linarith}
      \KeywordTok{done}
\NormalTok{    · }\CommentTok{{-}{-}case 2}
      \KeywordTok{exact}\NormalTok{ Nat.cast\_add\_one\_pos k}
      \KeywordTok{done}
\NormalTok{    · }\CommentTok{{-}{-}case 3}
\NormalTok{      convert add\_le\_add\_right h2 1}
      \KeywordTok{apply}\NormalTok{ Iff.intro}
\NormalTok{      · }\CommentTok{{-}{-}subcase 1}
        \KeywordTok{exact}\NormalTok{ fun a =\textgreater{} Nat.add\_le\_add\_right h2 1}
        \KeywordTok{done}
\NormalTok{      · }\CommentTok{{-}{-}subcase 2}
\NormalTok{        intro h14}
        \KeywordTok{apply}\NormalTok{ add\_le\_add\_right}
        \KeywordTok{exact}\NormalTok{ Iff.mpr Nat.cast\_le h2}
        \KeywordTok{done}
      \KeywordTok{done}
    \KeywordTok{done}
  \KeywordTok{have}\NormalTok{ h10 : 2 / (k + 1) \textless{} 2 / (2 / ε {-} 1 + 1) := }\KeywordTok{by}
    \KeywordTok{apply}\NormalTok{ div\_lt\_div\_of\_lt\_left}
\NormalTok{    · }\CommentTok{{-}{-}case 1}
      \KeywordTok{linarith}
      \KeywordTok{done}
\NormalTok{    · }\CommentTok{{-}{-}case 2}
      \KeywordTok{simp}
      \KeywordTok{apply}\NormalTok{ div\_pos}
      \KeywordTok{linarith}
      \KeywordTok{apply}\NormalTok{ h1}
      \KeywordTok{done}
\NormalTok{    · }\CommentTok{{-}{-}case 3}
\NormalTok{      convert add\_le\_add\_right h2 1}
      \KeywordTok{apply}\NormalTok{ Iff.intro}
\NormalTok{      · }\CommentTok{{-}{-}subcase 1}
\NormalTok{        intro h11}
        \KeywordTok{exact}\NormalTok{ Nat.add\_le\_add\_right h2 1}
        \KeywordTok{done}
\NormalTok{      · }\CommentTok{{-}{-}subcase 2}
\NormalTok{        intro h11}
        \KeywordTok{have}\NormalTok{ h12 : 2 / ε {-} 1 \textless{} (k : ℝ) := }\KeywordTok{by}
          \KeywordTok{simp}\NormalTok{ only []}
          \KeywordTok{apply}\NormalTok{ h13}
          \KeywordTok{done}
        \KeywordTok{exact}\NormalTok{ add\_lt\_add\_right h12 1}
        \KeywordTok{done}
      \KeywordTok{done}
    \KeywordTok{done}
  \KeywordTok{calc}
\NormalTok{    2 / (↑n + 1) ≤ (2 : ℝ) / (k + 1) := }\KeywordTok{by}
      \KeywordTok{apply}\NormalTok{ h9}
      \KeywordTok{done}
\NormalTok{    \_ \textless{} (2 : ℝ) / (2 / ε {-} 1 + 1) := }\KeywordTok{by}
      \KeywordTok{apply}\NormalTok{ h10}
      \KeywordTok{done}
\NormalTok{    \_ = ε := }\KeywordTok{by}
\NormalTok{      ring\_nf}
      \KeywordTok{apply}\NormalTok{ inv\_inv}
      \KeywordTok{done}
  \KeywordTok{done}
\end{Highlighting}
\end{Shaded}

\bookmarksetup{startatroot}

\hypertarget{conclusions}{%
\chapter{Conclusions}\label{conclusions}}

Ultimately lean is an incredibly powerful tool which does provide the
valuable proof verification benefits which make learning the language
worthwhile.

\bookmarksetup{startatroot}

\hypertarget{works-cited}{%
\chapter{Works Cited}\label{works-cited}}

This work had been formatted and styled from the book \emph{How To Prove
It With Lean}, written by Daniel J. Velleman. \emph{How To Prove It With
Lean} contains short excerpts from \emph{How To Prove It: A Structured
Approach, 3rd Edition}, by Daniel J. Velleman and published by Cambridge
University Press.

\bookmarksetup{startatroot}

\hypertarget{additional-space}{%
\chapter{Additional space}\label{additional-space}}

Extra chapter to write more things if needed!!



\end{document}
