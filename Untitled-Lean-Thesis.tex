% Options for packages loaded elsewhere
\PassOptionsToPackage{unicode}{hyperref}
\PassOptionsToPackage{hyphens}{url}
\PassOptionsToPackage{dvipsnames,svgnames,x11names}{xcolor}
%
\documentclass[
  letterpaper,
]{scrreprt}

\usepackage{amsmath,amssymb}
\usepackage{iftex}
\ifPDFTeX
  \usepackage[T1]{fontenc}
  \usepackage[utf8]{inputenc}
  \usepackage{textcomp} % provide euro and other symbols
\else % if luatex or xetex
  \usepackage{unicode-math}
  \defaultfontfeatures{Scale=MatchLowercase}
  \defaultfontfeatures[\rmfamily]{Ligatures=TeX,Scale=1}
\fi
\usepackage{lmodern}
\ifPDFTeX\else  
    % xetex/luatex font selection
  \setmonofont[]{JuliaMono}
\fi
% Use upquote if available, for straight quotes in verbatim environments
\IfFileExists{upquote.sty}{\usepackage{upquote}}{}
\IfFileExists{microtype.sty}{% use microtype if available
  \usepackage[]{microtype}
  \UseMicrotypeSet[protrusion]{basicmath} % disable protrusion for tt fonts
}{}
\makeatletter
\@ifundefined{KOMAClassName}{% if non-KOMA class
  \IfFileExists{parskip.sty}{%
    \usepackage{parskip}
  }{% else
    \setlength{\parindent}{0pt}
    \setlength{\parskip}{6pt plus 2pt minus 1pt}}
}{% if KOMA class
  \KOMAoptions{parskip=half}}
\makeatother
\usepackage{xcolor}
\setlength{\emergencystretch}{3em} % prevent overfull lines
\setcounter{secnumdepth}{0}
% Make \paragraph and \subparagraph free-standing
\ifx\paragraph\undefined\else
  \let\oldparagraph\paragraph
  \renewcommand{\paragraph}[1]{\oldparagraph{#1}\mbox{}}
\fi
\ifx\subparagraph\undefined\else
  \let\oldsubparagraph\subparagraph
  \renewcommand{\subparagraph}[1]{\oldsubparagraph{#1}\mbox{}}
\fi

\usepackage{color}
\usepackage{fancyvrb}
\newcommand{\VerbBar}{|}
\newcommand{\VERB}{\Verb[commandchars=\\\{\}]}
\DefineVerbatimEnvironment{Highlighting}{Verbatim}{commandchars=\\\{\}}
% Add ',fontsize=\small' for more characters per line
\usepackage{framed}
\definecolor{shadecolor}{RGB}{241,243,245}
\newenvironment{Shaded}{\begin{snugshade}}{\end{snugshade}}
\newcommand{\AlertTok}[1]{\textcolor[rgb]{0.68,0.00,0.00}{#1}}
\newcommand{\AnnotationTok}[1]{\textcolor[rgb]{0.37,0.37,0.37}{#1}}
\newcommand{\AttributeTok}[1]{\textcolor[rgb]{0.40,0.45,0.13}{#1}}
\newcommand{\BaseNTok}[1]{\textcolor[rgb]{0.68,0.00,0.00}{#1}}
\newcommand{\BuiltInTok}[1]{\textcolor[rgb]{0.00,0.23,0.31}{#1}}
\newcommand{\CharTok}[1]{\textcolor[rgb]{0.13,0.47,0.30}{#1}}
\newcommand{\CommentTok}[1]{\textcolor[rgb]{0.37,0.37,0.37}{#1}}
\newcommand{\CommentVarTok}[1]{\textcolor[rgb]{0.37,0.37,0.37}{\textit{#1}}}
\newcommand{\ConstantTok}[1]{\textcolor[rgb]{0.56,0.35,0.01}{#1}}
\newcommand{\ControlFlowTok}[1]{\textcolor[rgb]{0.00,0.23,0.31}{#1}}
\newcommand{\DataTypeTok}[1]{\textcolor[rgb]{0.68,0.00,0.00}{#1}}
\newcommand{\DecValTok}[1]{\textcolor[rgb]{0.68,0.00,0.00}{#1}}
\newcommand{\DocumentationTok}[1]{\textcolor[rgb]{0.37,0.37,0.37}{\textit{#1}}}
\newcommand{\ErrorTok}[1]{\textcolor[rgb]{0.68,0.00,0.00}{#1}}
\newcommand{\ExtensionTok}[1]{\textcolor[rgb]{0.00,0.23,0.31}{#1}}
\newcommand{\FloatTok}[1]{\textcolor[rgb]{0.68,0.00,0.00}{#1}}
\newcommand{\FunctionTok}[1]{\textcolor[rgb]{0.28,0.35,0.67}{#1}}
\newcommand{\ImportTok}[1]{\textcolor[rgb]{0.00,0.46,0.62}{#1}}
\newcommand{\InformationTok}[1]{\textcolor[rgb]{0.37,0.37,0.37}{#1}}
\newcommand{\KeywordTok}[1]{\textcolor[rgb]{0.00,0.23,0.31}{#1}}
\newcommand{\NormalTok}[1]{\textcolor[rgb]{0.00,0.23,0.31}{#1}}
\newcommand{\OperatorTok}[1]{\textcolor[rgb]{0.37,0.37,0.37}{#1}}
\newcommand{\OtherTok}[1]{\textcolor[rgb]{0.00,0.23,0.31}{#1}}
\newcommand{\PreprocessorTok}[1]{\textcolor[rgb]{0.68,0.00,0.00}{#1}}
\newcommand{\RegionMarkerTok}[1]{\textcolor[rgb]{0.00,0.23,0.31}{#1}}
\newcommand{\SpecialCharTok}[1]{\textcolor[rgb]{0.37,0.37,0.37}{#1}}
\newcommand{\SpecialStringTok}[1]{\textcolor[rgb]{0.13,0.47,0.30}{#1}}
\newcommand{\StringTok}[1]{\textcolor[rgb]{0.13,0.47,0.30}{#1}}
\newcommand{\VariableTok}[1]{\textcolor[rgb]{0.07,0.07,0.07}{#1}}
\newcommand{\VerbatimStringTok}[1]{\textcolor[rgb]{0.13,0.47,0.30}{#1}}
\newcommand{\WarningTok}[1]{\textcolor[rgb]{0.37,0.37,0.37}{\textit{#1}}}

\providecommand{\tightlist}{%
  \setlength{\itemsep}{0pt}\setlength{\parskip}{0pt}}\usepackage{longtable,booktabs,array}
\usepackage{calc} % for calculating minipage widths
% Correct order of tables after \paragraph or \subparagraph
\usepackage{etoolbox}
\makeatletter
\patchcmd\longtable{\par}{\if@noskipsec\mbox{}\fi\par}{}{}
\makeatother
% Allow footnotes in longtable head/foot
\IfFileExists{footnotehyper.sty}{\usepackage{footnotehyper}}{\usepackage{footnote}}
\makesavenoteenv{longtable}
\usepackage{graphicx}
\makeatletter
\def\maxwidth{\ifdim\Gin@nat@width>\linewidth\linewidth\else\Gin@nat@width\fi}
\def\maxheight{\ifdim\Gin@nat@height>\textheight\textheight\else\Gin@nat@height\fi}
\makeatother
% Scale images if necessary, so that they will not overflow the page
% margins by default, and it is still possible to overwrite the defaults
% using explicit options in \includegraphics[width, height, ...]{}
\setkeys{Gin}{width=\maxwidth,height=\maxheight,keepaspectratio}
% Set default figure placement to htbp
\makeatletter
\def\fps@figure{htbp}
\makeatother

\usepackage{mathrsfs}
\usepackage{twemojis}
\usepackage[normalem]{ulem}
\usepackage{scrlayer-scrpage}
\usepackage{textpos}
\automark{section}
\newcommand{\copyrightnotice}{\footnotesize\ Logan Johnson}
\newcommand{\longcopyrightnotice}{\footnotesize\ Logan Johnson 2023}
\setkomafont{pagefoot}{\upshape}
\ifoot*{\longcopyrightnotice}
\cfoot*{}
\ofoot*{\pagemark}

\makeatletter
\def\redsquiggly{\bgroup \markoverwith{\textcolor{red}{\lower3.5\p@\hbox{\sixly \char58}}}\ULon}
\def\brownsquiggly{\bgroup \markoverwith{\textcolor[HTML]{B8860B}{\lower3.5\p@\hbox{\sixly \char58}}}\ULon}
\def\bluesquiggly{\bgroup \markoverwith{\textcolor[HTML]{1E90FF}{\lower3.5\p@\hbox{\sixly \char58}}}\ULon}
\makeatother

\renewcommand{\NormalTok}[1]{\textcolor[HTML]{000000}{#1}}
\renewcommand{\KeywordTok}[1]{\textcolor[HTML]{0000FF}{#1}}
\renewcommand{\SpecialCharTok}[1]{}
\renewcommand{\ErrorTok}[1]{\redsquiggly{#1}}
\renewcommand{\WarningTok}[1]{\redsquiggly{\textcolor[HTML]{0000FF}{#1}}}
\renewcommand{\StringTok}[1]{\textcolor[HTML]{A52A2A}{#1}}
\renewcommand{\CommentTok}[1]{\textcolor[HTML]{008000}{#1}}
\renewcommand{\InformationTok}[1]{\textcolor[HTML]{D2691E}{\textbf{#1}}}
\renewcommand{\RegionMarkerTok}[1]{▼\:\textcolor[HTML]{008000}{\textbf{#1}}}
\renewcommand{\SpecialStringTok}[1]{\textcolor[HTML]{4682B4}{\textbf{#1}}}
\renewcommand{\ConstantTok}[1]{\textcolor[HTML]{DC143C}{#1}}
\renewcommand{\AnnotationTok}[1]{\brownsquiggly{#1}}
\renewcommand{\AlertTok}[1]{\brownsquiggly{\textcolor[HTML]{0000FF}{#1}}}
\renewcommand{\OtherTok}[1]{\bluesquiggly{#1}}
\renewcommand{\DocumentationTok}[1]{\bluesquiggly{\textcolor[HTML]{0000FF}{#1}}}

\newenvironment{ind}
	{\begin{list}{}{\setlength{\leftmargin}{1em}}\item\relax}
	{\end{list}}

%redefines Shaded so it can't break
\newcommand{\nobreakShaded}{\renewenvironment{Shaded}
	{\begin{tcolorbox}[frame hidden, enhanced, interior hidden, boxrule=0pt,
		borderline west={3pt}{0pt}{shadecolor}, sharp corners]}
	{\end{tcolorbox}}}

%Make end of environment ignore pars that come after it
\def\useignorespacesandallpars#1\ignorespaces\fi{%
#1\fi\ignorespacesandallpars}

\makeatletter
\def\ignorespacesandallpars{%
  \@ifnextchar\par
    {\expandafter\ignorespacesandallpars\@gobble}%
    {}%
}
\makeatother

\newenvironment{inpt}
	{\nobreakShaded\noindent\begin{minipage}[t]{0.63\textwidth}
		\uline{Lean File}}
	{\end{minipage}\hfill\useignorespacesandallpars}

\newenvironment{outpt}
	{\nobreakShaded\begin{minipage}[t]{0.32\textwidth}
		\uline{Tactic State in Infoview}}
	{\end{minipage}}

\newenvironment{bef}
	{\nobreakShaded\noindent\begin{minipage}[t]{0.475\textwidth}
		\uline{Tactic State Before Using Strategy}}
	{\end{minipage}\hfill\useignorespacesandallpars}

\newenvironment{aft}
	{\nobreakShaded\begin{minipage}[t]{0.475\textwidth}
		\uline{Tactic State After Using Strategy}}
	{\end{minipage}}

\newenvironment{numex}[1]
	{\begin{minipage}[t]{0.04\textwidth}\vspace{8pt}{#1}.
		\end{minipage}\nobreakShaded\begin{minipage}[t]{0.96\textwidth}\vspace{0pt}}
	{\end{minipage}}

\newenvironment{mdsk}
	{\medskip}
	{}

\newenvironment{absnobreak}
  {\par\nobreak\vfil\penalty0\vfilneg
   \vtop\bgroup}
  {\par\xdef\tpd{\the\prevdepth}\egroup
   \prevdepth=\tpd}

\newcommand{\excl}[1]{}
\newcommand{\incl}[1]{#1}

\newcommand{\setmin}{\mathbin{\backslash}}
\newcommand{\symmdiff}{\bigtriangleup}

\pagenumbering{roman}  %So front matter uses roman numerals.  Switch back to arabic at beginning of preface.
\publishers{\longcopyrightnotice}
\makeatletter
\makeatother
\makeatletter
\@ifpackageloaded{bookmark}{}{\usepackage{bookmark}}
\makeatother
\makeatletter
\@ifpackageloaded{caption}{}{\usepackage{caption}}
\AtBeginDocument{%
\ifdefined\contentsname
  \renewcommand*\contentsname{Table of contents}
\else
  \newcommand\contentsname{Table of contents}
\fi
\ifdefined\listfigurename
  \renewcommand*\listfigurename{List of Figures}
\else
  \newcommand\listfigurename{List of Figures}
\fi
\ifdefined\listtablename
  \renewcommand*\listtablename{List of Tables}
\else
  \newcommand\listtablename{List of Tables}
\fi
\ifdefined\figurename
  \renewcommand*\figurename{Figure}
\else
  \newcommand\figurename{Figure}
\fi
\ifdefined\tablename
  \renewcommand*\tablename{Table}
\else
  \newcommand\tablename{Table}
\fi
}
\@ifpackageloaded{float}{}{\usepackage{float}}
\floatstyle{ruled}
\@ifundefined{c@chapter}{\newfloat{codelisting}{h}{lop}}{\newfloat{codelisting}{h}{lop}[chapter]}
\floatname{codelisting}{Listing}
\newcommand*\listoflistings{\listof{codelisting}{List of Listings}}
\usepackage{amsthm}
\theoremstyle{remark}
\AtBeginDocument{\renewcommand*{\proofname}{Proof}}
\newtheorem*{remark}{Remark}
\newtheorem*{solution}{Solution}
\makeatother
\makeatletter
\@ifpackageloaded{caption}{}{\usepackage{caption}}
\@ifpackageloaded{subcaption}{}{\usepackage{subcaption}}
\makeatother
\makeatletter
\@ifpackageloaded{tcolorbox}{}{\usepackage[skins,breakable]{tcolorbox}}
\makeatother
\makeatletter
\@ifundefined{shadecolor}{\definecolor{shadecolor}{rgb}{.97, .97, .97}}
\makeatother
\makeatletter
\makeatother
\makeatletter
\makeatother
\ifLuaTeX
  \usepackage{selnolig}  % disable illegal ligatures
\fi
\IfFileExists{bookmark.sty}{\usepackage{bookmark}}{\usepackage{hyperref}}
\IfFileExists{xurl.sty}{\usepackage{xurl}}{} % add URL line breaks if available
\urlstyle{same} % disable monospaced font for URLs
\hypersetup{
  pdftitle={Untitled Lean Thesis},
  pdfauthor={Logan Johnson},
  colorlinks=true,
  linkcolor={blue},
  filecolor={Maroon},
  citecolor={Blue},
  urlcolor={Blue},
  pdfcreator={LaTeX via pandoc}}

\title{Untitled Lean Thesis}
\author{Logan Johnson}
\date{Invalid Date}

\begin{document}
\newcommand{\fromtop}[1]{%
  \dimexpr-1in-\topskip-\topmargin-\headheight-\headsep+#1\relax
}
\newcommand{\fromleft}[1]{%
  \dimexpr-1in-\oddsidemargin+#1\relax
}
\setlength{\textwidth}{6in}
\begin{titlepage}
\cleardoublepage
\thispagestyle{empty}
\begin{textblock*}{\paperwidth}(\fromleft{0cm},\fromtop{144pt})
\centering
UNTITLED LEAN THESIS
\end{textblock*}
\begin{textblock*}{\paperwidth}(\fromleft{0cm},\fromtop{300pt})
\centering
A Thesis Submitted to the Faculty of\\
\vspace{12pt}
Georgetown College\\
\vspace{12pt}
In Partial Fulfillment of the Requirements for the\\
\vspace{12pt}
Honors Program
\end{textblock*}
\begin{textblock*}{\paperwidth}(\fromleft{0cm},\fromtop{492pt})
\centering
By\\
\vspace{12pt}
Logan Johnson
\end{textblock*}
\begin{textblock*}{\paperwidth}(\fromleft{0cm},\fromtop{600pt})
\centering
Georgetown, Kentucky\\
\vspace{12pt}
May 2024
\end{textblock*}
\pagebreak
\thispagestyle{empty}
\begin{textblock*}{\paperwidth}(\fromleft{0cm},\fromtop{180pt})
\centering
Abstract\\
\vspace{12pt}
UNTITLED LEAN THESIS\\
\vspace{12pt}
Logan C. Johnson\\
\vspace{12pt}
idk dr burch?? maybe dr white???
\end{textblock*}
\begin{textblock*}{\textwidth}(\fromleft{1.5in},\fromtop{300pt})
\raggedright
I'm fr out here putting words on a page . . . they'll really just let anyone write a thesis these days. Here is the text of your abstract. It goes on and on and on. It goes on like this for about
150 words, so it should all fit on this page. Note that the Abstract comes before the title page and
has no page number. The rest of this paragraph is a filler. It goes on like this for about 150
words, so it should all fit on this page. Note that the Abstract comes before the title page and has
no page number. It goes on like this for about 150 words, so it should all fit on this page. Note
that the Abstract comes before the title page and has no page number. It goes on like this for
about 150 words, so it should all fit on this page. Note that the Abstract comes before the title
page and has no page number. It goes on like this for about 150 words, so it should all fit on this
page. Note that the Abstract comes before the title page and has no page number. It goes on like
this for about 150 words, so it should all fit on this page. Note that the Abstract comes before the
title page and has no page number. It goes on like this for about 150 words, so it should all fit on
this page. Note that the Abstract comes before the title page and has no page number. If your
abstract is more than 250 words, consider shortening it.
\end{textblock*}
\pagebreak
\thispagestyle{empty}
\begin{textblock*}{\paperwidth}(\fromleft{2in},\fromtop{144pt})
APPROVED BY THE DIRECTOR OF HONORS THESES:
\end{textblock*}
\begin{textblock*}{\paperwidth}(\fromleft{3in},\fromtop{192pt})
\underline{\hspace{4.5in}}
\end{textblock*}
\begin{textblock*}{\paperwidth}(\fromleft{3in},\fromtop{216pt})
Dr. Homer White, Department of Mathematics
\end{textblock*}
\begin{textblock*}{\paperwidth}(\fromleft{2in},\fromtop{396pt})
APPROVED BY THE HONORS PROGRAM:
\end{textblock*}
\begin{textblock*}{\paperwidth}(\fromleft{2in},\fromtop{444pt})
\underline{\hspace{4.5in}}
\end{textblock*}
\begin{textblock*}{\paperwidth}(\fromleft{2in},\fromtop{468pt})
Dr. Barbara Burch, Director
\end{textblock*}
\begin{textblock*}{\paperwidth}(\fromleft{1.5in},\fromtop{612pt})
DATE:\underline{\hspace{2in}}
\end{textblock*}
\end{titlepage}%Can't be in preamble because Quarto loads amsthm too late.
\theoremstyle{plain}
\newtheorem*{thm}{Theorem}
\newcommand{\thmnm}{Theorem}
\newtheorem*{namedthm}{\thmnm}
\theoremstyle{definition}
\newtheorem*{dfn}{Definition}
\newcommand{\defnm}{Definition}
\newtheorem*{nameddfn}{\defnm}

\newenvironment{nthm}[1]
  {\renewcommand{\thmnm}{#1}\begin{namedthm}}
  {\end{namedthm}}

\newenvironment{ndfn}[1]
  {\renewcommand{\defnm}{#1}\begin{nameddfn}}
  {\end{nameddfn}}

\newenvironment{npf}[1]
  {\begin{proof}[#1]}
  {\end{proof}}

% Usage:
% ::: {.thm} will create an unnumbered thm environment with title Theorem.
% ::: {.nthm arguments="Name of Theorem"} will create an unnumbered theorem whose title is Name of Theorem.

\let\oldgreater\textgreater
\renewcommand{\textgreater}{\null\oldgreater}   % To prevent => changing to double arrow

\ifdefined\Shaded\renewenvironment{Shaded}{\begin{tcolorbox}[interior hidden, frame hidden, boxrule=0pt, enhanced, borderline west={3pt}{0pt}{shadecolor}, breakable, sharp corners]}{\end{tcolorbox}}\fi

\renewcommand*\contentsname{Table of contents}
{
\hypersetup{linkcolor=}
\setcounter{tocdepth}{1}
\tableofcontents
}
\bookmarksetup{startatroot}

\hypertarget{preface}{%
\chapter*{Preface}\label{preface}}
\addcontentsline{toc}{chapter}{Preface}

\markboth{Preface}{Preface}

\pagenumbering{arabic}
\markdouble{Preface}

I will not do my homework today.

\begin{Shaded}
\begin{Highlighting}[]
\FunctionTok{sum}\NormalTok{(}\DecValTok{4}\NormalTok{, }\DecValTok{7}\NormalTok{, }\DecValTok{3}\NormalTok{)}
\end{Highlighting}
\end{Shaded}

\begin{verbatim}
[1] 14
\end{verbatim}

Hello World

\hypertarget{making-chapters}{%
\section*{Making Chapters}\label{making-chapters}}
\addcontentsline{toc}{section}{Making Chapters}

\markright{Making Chapters}

I am using this section to figure out how to incorporate a table of
contents and different sections/chapters of the paper. This should prove
useful in the final thesis and allow readers to quickly jump to
important or interesting sections.

\hypertarget{incorporating-some-code}{%
\section*{Incorporating Some Code}\label{incorporating-some-code}}
\addcontentsline{toc}{section}{Incorporating Some Code}

\markright{Incorporating Some Code}

I will also be able to use some LaTeX equations within the document
which could halp to make the paper look quite nice.

If \(a < b\) and \(c \le d\), prove that \(a + c \le b + d\). It just so
happens that I was able to prove this using lean!

\begin{verbatim}
example (a b c d : ℝ) (h1: a < b) (h2 : c ≤ d) : a + c < b + d := by
  by_cases h3 : c = d
  rw [h3]
  apply add_lt_add_right h1
  push_neg at h3
  have h4 : c < d := by
    apply Ne.lt_of_le h3 h2
  apply add_lt_add h1 h4
  done
\end{verbatim}

Now to display the benefits of the lean infoview!

\hypertarget{showing-the-infoview-with-picture-sequences}{%
\section*{Showing the Infoview with Picture
Sequences}\label{showing-the-infoview-with-picture-sequences}}
\addcontentsline{toc}{section}{Showing the Infoview with Picture
Sequences}

\markright{Showing the Infoview with Picture Sequences}

Not going to do this now as I think the columns are far superior.

\hypertarget{showing-infoview-with-columns}{%
\section*{Showing Infoview with
Columns}\label{showing-infoview-with-columns}}
\addcontentsline{toc}{section}{Showing Infoview with Columns}

\markright{Showing Infoview with Columns}

As we can clearly see, this is the first step of the code and it can now
be explained with great ease. Now onto the next step!

\begin{inpt}

\begin{Shaded}
\begin{Highlighting}[]
\KeywordTok{theorem}\NormalTok{ Example\_3\_2\_4\_v2 (P Q R : }\KeywordTok{Prop}\NormalTok{)}
\NormalTok{    (h : P → (Q → R)) : ¬R → (P → ¬Q) := }\KeywordTok{by}
  \KeywordTok{assume}\NormalTok{ h2 : ¬R}
  \KeywordTok{assume}\NormalTok{ h3 : P}
  \SpecialCharTok{**}\WarningTok{done}\SpecialCharTok{::}
\end{Highlighting}
\end{Shaded}

\end{inpt}

\begin{outpt}

\begin{Shaded}
\begin{Highlighting}[]
\InformationTok{P Q R }\NormalTok{: Prop}
\InformationTok{h }\NormalTok{: P → Q → R}
\InformationTok{h2 }\NormalTok{: ¬R}
\InformationTok{h3 }\NormalTok{: P}
\NormalTok{⊢ ¬Q}
\end{Highlighting}
\end{Shaded}

\end{outpt}

Click on the \emph{Extensions} icon on the left side of the window,
which is circled in red in the image above. That will bring up a list of
available extensions:

\includegraphics{Images/FindExtension.png}

\hypertarget{acknowledgments}{%
\section*{Acknowledgments}\label{acknowledgments}}
\addcontentsline{toc}{section}{Acknowledgments}

\markright{Acknowledgments}

\bookmarksetup{startatroot}

\hypertarget{real-analysis}{%
\chapter{Real Analysis}\label{real-analysis}}

\begin{longtable}[]{@{}cc@{}}
\toprule\noalign{}
Symbol & Meaning \\
\midrule\noalign{}
\endhead
\bottomrule\noalign{}
\endlastfoot
\(\neg\) & not \\
\(\wedge\) & and \\
\(\vee\) & or \\
\(\to\) & if \ldots{} then \\
\(\leftrightarrow\) & iff (that is, if and only if) \\
\end{longtable}

\hypertarget{prop-laws}{}
\begin{longtable}[]{@{}
  >{\raggedright\arraybackslash}p{(\columnwidth - 6\tabcolsep) * \real{0.3226}}
  >{\centering\arraybackslash}p{(\columnwidth - 6\tabcolsep) * \real{0.2258}}
  >{\centering\arraybackslash}p{(\columnwidth - 6\tabcolsep) * \real{0.2258}}
  >{\centering\arraybackslash}p{(\columnwidth - 6\tabcolsep) * \real{0.2258}}@{}}
\toprule\noalign{}
\begin{minipage}[b]{\linewidth}\raggedright
Name
\end{minipage} & \begin{minipage}[b]{\linewidth}\centering
\end{minipage} & \begin{minipage}[b]{\linewidth}\centering
Equivalence
\end{minipage} & \begin{minipage}[b]{\linewidth}\centering
\end{minipage} \\
\midrule\noalign{}
\endhead
\bottomrule\noalign{}
\endlastfoot
De Morgan's Laws & \(\neg (P \wedge Q)\) & is equivalent to &
\(\neg P \vee \neg Q\) \\
& \(\neg (P \vee Q)\) & is equivalent to & \(\neg P \wedge \neg Q\) \\
Double Negation Law & \(\neg\neg P\) & is equivalent to & \(P\) \\
Conditional Laws & \(P \to Q\) & is equivalent to & \(\neg P \vee Q\) \\
& \(P \to Q\) & is equivalent to & \(\neg(P \wedge \neg Q)\) \\
Contrapositive Law & \(P \to Q\) & is equivalent to &
\(\neg Q \to \neg P\) \\
\end{longtable}

\begin{quote}
\(A \cap B = \{x \mid x \in A \wedge x \in B\} = {}\) the
\emph{intersection} of \(A\) and \(B\),

\(A \cup B = \{x \mid x \in A \vee x \in B\} = {}\) the \emph{union} of
\(A\) and \(B\),

\(A \setmin B = \{x \mid x \in A \wedge x \notin B\} = {}\) the
\emph{difference} of \(A\) and \(B\),

\(A \symmdiff B = (A \setmin B) \cup (B \setmin A) = {}\) the
\emph{symmetric difference} of \(A\) and \(B\).

\end{quote}

\bookmarksetup{startatroot}

\hypertarget{functional-programming}{%
\chapter{Functional Programming}\label{functional-programming}}

\begin{quote}
\(\forall x\,P(x)\) means ``for all \(x\), \(P(x)\),''

\end{quote}

\begin{longtable}[]{@{}ccc@{}}
\toprule\noalign{}
& Quantifier Negation Laws & \\
\midrule\noalign{}
\endhead
\bottomrule\noalign{}
\endlastfoot
\(\neg \exists x\,P(x)\) & is equivalent to &
\(\forall x\,\neg P(x)\) \\
\(\neg \forall x\,P(x)\) & is equivalent to &
\(\exists x\,\neg P(x)\) \\
\end{longtable}

\bookmarksetup{startatroot}

\hypertarget{lean-as-a-theorem-prover}{%
\chapter{Lean as a Theorem Prover}\label{lean-as-a-theorem-prover}}

\hypertarget{inequality-addition}{%
\section{Inequality Addition}\label{inequality-addition}}

Step 1 Putting the initial theorem we want to show in lean code and
observing the final goal in the infoview.

\begin{inpt}

\begin{Shaded}
\begin{Highlighting}[]
\KeywordTok{example}\NormalTok{ (a b c d : ℝ) (h1: a \textless{} b) }
\NormalTok{    (h2 : c ≤ d) : a + c \textless{} b + d := }\KeywordTok{by}
  
  \SpecialCharTok{**}\WarningTok{done}\SpecialCharTok{::}
\end{Highlighting}
\end{Shaded}

\end{inpt}

\begin{outpt}

\begin{Shaded}
\begin{Highlighting}[]
\InformationTok{R}\NormalTok{: Type u\_1}
\InformationTok{inst✝}\NormalTok{: Ring R}
\InformationTok{abcd}\NormalTok{: ℝ}
\InformationTok{h1}\NormalTok{: a \textless{} b}
\InformationTok{h2}\NormalTok{: c ≤ d}
\NormalTok{⊢ a + c \textless{} b + d}
\end{Highlighting}
\end{Shaded}

\end{outpt}

Step 2

\begin{inpt}

\begin{Shaded}
\begin{Highlighting}[]
\KeywordTok{example}\NormalTok{ (a b c d : ℝ) (h1: a \textless{} b) }
\NormalTok{    (h2 : c ≤ d) : a + c \textless{} b + d := }\KeywordTok{by}
  \KeywordTok{by\_cases}\NormalTok{ h3 : c = d}

  \SpecialCharTok{**}\WarningTok{done}\SpecialCharTok{::}
\end{Highlighting}
\end{Shaded}

\end{inpt}

\begin{outpt}

\begin{Shaded}
\begin{Highlighting}[]
\InformationTok{R}\NormalTok{: Type u\_1}
\InformationTok{inst✝}\NormalTok{: Ring R}
\InformationTok{abcd}\NormalTok{: ℝ}
\InformationTok{h1}\NormalTok{: a \textless{} b}
\InformationTok{h2}\NormalTok{: c ≤ d}
\InformationTok{h3}\NormalTok{: c = d}
\NormalTok{⊢ a + c \textless{} b + d}
\end{Highlighting}
\end{Shaded}

\end{outpt}

Step 3

\begin{inpt}

\begin{Shaded}
\begin{Highlighting}[]
\KeywordTok{example}\NormalTok{ (a b c d : ℝ) (h1: a \textless{} b) }
\NormalTok{    (h2 : c ≤ d) : a + c \textless{} b + d := }\KeywordTok{by}
  \KeywordTok{by\_cases}\NormalTok{ h3 : c = d}
  \KeywordTok{rw}\NormalTok{ [h3]}

  \SpecialCharTok{**}\WarningTok{done}\SpecialCharTok{::}
\end{Highlighting}
\end{Shaded}

\end{inpt}

\begin{outpt}

\begin{Shaded}
\begin{Highlighting}[]
\InformationTok{R}\NormalTok{: Type u\_1}
\InformationTok{inst✝}\NormalTok{: Ring R}
\InformationTok{abcd}\NormalTok{: ℝ}
\InformationTok{h1}\NormalTok{: a \textless{} b}
\InformationTok{h2}\NormalTok{: c ≤ d}
\InformationTok{h3}\NormalTok{: c = d}
\NormalTok{⊢ a + d \textless{} b + d}
\end{Highlighting}
\end{Shaded}

\end{outpt}

Step 4

\begin{inpt}

\begin{Shaded}
\begin{Highlighting}[]
\KeywordTok{example}\NormalTok{ (a b c d : ℝ) (h1: a \textless{} b) }
\NormalTok{    (h2 : c ≤ d) : a + c \textless{} b + d := }\KeywordTok{by}
  \KeywordTok{by\_cases}\NormalTok{ h3 : c = d}
  \KeywordTok{rw}\NormalTok{ [h3]}
  \KeywordTok{apply}\NormalTok{ add\_lt\_add\_right h1}

  \SpecialCharTok{**}\WarningTok{done}\SpecialCharTok{::}
\end{Highlighting}
\end{Shaded}

\end{inpt}

\begin{outpt}

\begin{Shaded}
\begin{Highlighting}[]
\InformationTok{R}\NormalTok{: Type u\_1}
\InformationTok{inst✝}\NormalTok{: Ring R}
\InformationTok{abcd}\NormalTok{: ℝ}
\InformationTok{h1}\NormalTok{: a \textless{} b}
\InformationTok{h2}\NormalTok{: c ≤ d}
\InformationTok{h3}\NormalTok{: ¬c = d}
\NormalTok{⊢ a + c \textless{} b + d}
\end{Highlighting}
\end{Shaded}

\end{outpt}

Step 5

\begin{inpt}

\begin{Shaded}
\begin{Highlighting}[]
\KeywordTok{example}\NormalTok{ (a b c d : ℝ) (h1: a \textless{} b) }
\NormalTok{    (h2 : c ≤ d) : a + c \textless{} b + d := }\KeywordTok{by}
  \KeywordTok{by\_cases}\NormalTok{ h3 : c = d}
  \KeywordTok{rw}\NormalTok{ [h3]}
  \KeywordTok{apply}\NormalTok{ add\_lt\_add\_right h1}
  \KeywordTok{push\_neg} \KeywordTok{at}\NormalTok{ h3}

  \SpecialCharTok{**}\WarningTok{done}\SpecialCharTok{::}
\end{Highlighting}
\end{Shaded}

\end{inpt}

\begin{outpt}

\begin{Shaded}
\begin{Highlighting}[]
\InformationTok{R}\NormalTok{: Type u\_1}
\InformationTok{inst✝}\NormalTok{: Ring R}
\InformationTok{abcd}\NormalTok{: ℝ}
\InformationTok{h1}\NormalTok{: a \textless{} b}
\InformationTok{h2}\NormalTok{: c ≤ d}
\InformationTok{h3}\NormalTok{: c ≠ d}
\NormalTok{⊢ a + c \textless{} b + d}
\end{Highlighting}
\end{Shaded}

\end{outpt}

Step 6a We can see here that the lean infoview is now displaying my new
hypothesis as the current goal.

\begin{inpt}

\begin{Shaded}
\begin{Highlighting}[]
\KeywordTok{example}\NormalTok{ (a b c d : ℝ) (h1: a \textless{} b) }
\NormalTok{    (h2 : c ≤ d) : a + c \textless{} b + d := }\KeywordTok{by}
  \KeywordTok{by\_cases}\NormalTok{ h3 : c = d}
  \KeywordTok{rw}\NormalTok{ [h3]}
  \KeywordTok{apply}\NormalTok{ add\_lt\_add\_right h1}
  \KeywordTok{push\_neg} \KeywordTok{at}\NormalTok{ h3}
  \KeywordTok{have}\NormalTok{ h4 : c \textless{} d := }\KeywordTok{by}

  \SpecialCharTok{**}\WarningTok{done}\SpecialCharTok{::}
\end{Highlighting}
\end{Shaded}

\end{inpt}

\begin{outpt}

\begin{Shaded}
\begin{Highlighting}[]
\InformationTok{R}\NormalTok{: Type u\_1}
\InformationTok{inst✝}\NormalTok{: Ring R}
\InformationTok{abcd}\NormalTok{: ℝ}
\InformationTok{h1}\NormalTok{: a \textless{} b}
\InformationTok{h2}\NormalTok{: c ≤ d}
\InformationTok{h3}\NormalTok{: c ≠ d}
\NormalTok{⊢ c \textless{} d}
\end{Highlighting}
\end{Shaded}

\end{outpt}

Step 6b

\begin{inpt}

\begin{Shaded}
\begin{Highlighting}[]
\KeywordTok{example}\NormalTok{ (a b c d : ℝ) (h1: a \textless{} b) }
\NormalTok{    (h2 : c ≤ d) : a + c \textless{} b + d := }\KeywordTok{by}
  \KeywordTok{by\_cases}\NormalTok{ h3 : c = d}
  \KeywordTok{rw}\NormalTok{ [h3]}
  \KeywordTok{apply}\NormalTok{ add\_lt\_add\_right h1}
  \KeywordTok{push\_neg} \KeywordTok{at}\NormalTok{ h3}
  \KeywordTok{have}\NormalTok{ h4 : c \textless{} d := }\KeywordTok{by}
    \KeywordTok{apply}\NormalTok{ Ne.lt\_of\_le h3 h2}

  \SpecialCharTok{**}\WarningTok{done}\SpecialCharTok{::}
\end{Highlighting}
\end{Shaded}

\end{inpt}

\begin{outpt}

\begin{Shaded}
\begin{Highlighting}[]
\SpecialStringTok{No}\InformationTok{ }\SpecialStringTok{goals}
\end{Highlighting}
\end{Shaded}

\end{outpt}

Step 7

\begin{inpt}

\begin{Shaded}
\begin{Highlighting}[]
\KeywordTok{example}\NormalTok{ (a b c d : ℝ) (h1: a \textless{} b) }
\NormalTok{    (h2 : c ≤ d) : a + c \textless{} b + d := }\KeywordTok{by}
  \KeywordTok{by\_cases}\NormalTok{ h3 : c = d}
  \KeywordTok{rw}\NormalTok{ [h3]}
  \KeywordTok{apply}\NormalTok{ add\_lt\_add\_right h1}
  \KeywordTok{push\_neg} \KeywordTok{at}\NormalTok{ h3}
  \KeywordTok{have}\NormalTok{ h4 : c \textless{} d := }\KeywordTok{by}
    \KeywordTok{apply}\NormalTok{ Ne.lt\_of\_le h3 h2}
  \KeywordTok{apply}\NormalTok{ add\_lt\_add h1 h4}
  \KeywordTok{done}
\end{Highlighting}
\end{Shaded}

\end{inpt}

\begin{outpt}

\begin{Shaded}
\begin{Highlighting}[]
\SpecialStringTok{No}\InformationTok{ }\SpecialStringTok{goals}
\end{Highlighting}
\end{Shaded}

\end{outpt}

\hypertarget{to-prove-a-goal-of-the-form-p-q}{%
\subsubsection{\texorpdfstring{To prove a goal of the form
\texttt{P\ →\ Q}:}{To prove a goal of the form P → Q:}}\label{to-prove-a-goal-of-the-form-p-q}}

\begin{enumerate}
\def\labelenumi{\arabic{enumi}.}
\tightlist
\item
  Assume \texttt{P} is true and prove \texttt{Q}.
\item
  Assume \texttt{Q} is false and prove that \texttt{P} is false.
\end{enumerate}

\begin{bef}

\begin{Shaded}
\begin{Highlighting}[]
\SpecialCharTok{\textgreater{}\textgreater{}}\NormalTok{ ⋮}
\NormalTok{⊢ P → Q}
\end{Highlighting}
\end{Shaded}

\end{bef}

\begin{aft}

\begin{Shaded}
\begin{Highlighting}[]
\SpecialCharTok{\textgreater{}\textgreater{}}\NormalTok{ ⋮}
\InformationTok{h }\NormalTok{: P}
\NormalTok{⊢ Q}
\end{Highlighting}
\end{Shaded}

\end{aft}

\begin{longtable}[]{@{}ccrcl@{}}
\toprule\noalign{}
\texttt{Q} & \texttt{¬Q} & \texttt{(Q} & \texttt{→} & \texttt{False)} \\
\midrule\noalign{}
\endhead
\bottomrule\noalign{}
\endlastfoot
F & T & F & T & ~~\excl{~} F \\
T & F & T & F & ~~\excl{~} F \\
\end{longtable}

\begin{inpt}

\begin{Shaded}
\begin{Highlighting}[]
\KeywordTok{theorem}\NormalTok{ Example\_3\_2\_4\_v2 (P Q R : }\KeywordTok{Prop}\NormalTok{)}
\NormalTok{    (h : P → (Q → R)) : ¬R → (P → ¬Q) := }\KeywordTok{by}
  \KeywordTok{assume}\NormalTok{ h2 : ¬R}
  \KeywordTok{assume}\NormalTok{ h3 : P}
  \SpecialCharTok{**}\WarningTok{done}\SpecialCharTok{::}
\end{Highlighting}
\end{Shaded}

\end{inpt}

\begin{outpt}

\begin{Shaded}
\begin{Highlighting}[]
\InformationTok{P Q R }\NormalTok{: Prop}
\InformationTok{h }\NormalTok{: P → Q → R}
\InformationTok{h2 }\NormalTok{: ¬R}
\InformationTok{h3 }\NormalTok{: P}
\NormalTok{⊢ ¬Q}
\end{Highlighting}
\end{Shaded}

\end{outpt}

\hypertarget{exercises}{%
\subsection{Exercises}\label{exercises}}

Fill in proofs of the following theorems. All of them are based on
exercises in \emph{HTPI}.

\begin{numex}{1}

\begin{Shaded}
\begin{Highlighting}[]
\KeywordTok{theorem}\NormalTok{ Exercise\_3\_2\_1a (P Q R : }\KeywordTok{Prop}\NormalTok{)}
\NormalTok{    (h1 : P → Q) (h2 : Q → R) : P → R := }\KeywordTok{by}
  
  \SpecialCharTok{**}\WarningTok{done}\SpecialCharTok{::}
\end{Highlighting}
\end{Shaded}

\end{numex}

\begin{numex}{2}

\begin{Shaded}
\begin{Highlighting}[]
\KeywordTok{theorem}\NormalTok{ Exercise\_3\_2\_1b (P Q R : }\KeywordTok{Prop}\NormalTok{)}
\NormalTok{    (h1 : ¬R → (P → ¬Q)) : P → (Q → R) := }\KeywordTok{by}
  
  \SpecialCharTok{**}\WarningTok{done}\SpecialCharTok{::}
\end{Highlighting}
\end{Shaded}

\end{numex}

\begin{numex}{3}

\begin{Shaded}
\begin{Highlighting}[]
\KeywordTok{theorem}\NormalTok{ Exercise\_3\_2\_2a (P Q R : }\KeywordTok{Prop}\NormalTok{)}
\NormalTok{    (h1 : P → Q) (h2 : R → ¬Q) : P → ¬R := }\KeywordTok{by}
  
  \SpecialCharTok{**}\WarningTok{done}\SpecialCharTok{::}
\end{Highlighting}
\end{Shaded}

\end{numex}

\begin{numex}{4}

\begin{Shaded}
\begin{Highlighting}[]
\KeywordTok{theorem}\NormalTok{ Exercise\_3\_2\_2b (P Q : }\KeywordTok{Prop}\NormalTok{)}
\NormalTok{    (h1 : P) : Q → ¬(Q → ¬P) := }\KeywordTok{by}
  
  \SpecialCharTok{**}\WarningTok{done}\SpecialCharTok{::}
\end{Highlighting}
\end{Shaded}

\end{numex}

\hypertarget{proofs-involving-quantifiers}{%
\section{3.3. Proofs Involving
Quantifiers}\label{proofs-involving-quantifiers}}

\begin{ind}
Let \texttt{x} stand for an arbitrary object of type \texttt{U} and
prove \texttt{P\ x}. If the letter \texttt{x} is already being used in
the proof to stand for something, then you must choose an unused
variable, say \texttt{y}, to stand for the arbitrary object, and prove
\texttt{P\ y}.

\end{ind}

\begin{longtable}[]{@{}ccc@{}}
\toprule\noalign{}
& \texttt{quant\_neg} Tactic & \\
\midrule\noalign{}
\endhead
\bottomrule\noalign{}
\endlastfoot
\texttt{¬∀\ (x\ :\ U),\ P\ x} & is changed to &
\texttt{∃\ (x\ :\ U),\ ¬P\ x} \\
\texttt{¬∃\ (x\ :\ U),\ P\ x} & is changed to &
\texttt{∀\ (x\ :\ U),\ ¬P\ x} \\
\texttt{∀\ (x\ :\ U),\ P\ x} & is changed to &
\texttt{¬∃\ (x\ :\ U),\ ¬P\ x} \\
\texttt{∃\ (x\ :\ U),\ P\ x} & is changed to &
\texttt{¬∀\ (x\ :\ U),\ ¬P\ x} \\
\end{longtable}

\begin{thm}
Suppose \(B\) is a set and \(\mathcal{F}\) is a family of sets. If
\(\bigcup\mathcal{F} \subseteq B\) then
\(\mathcal{F} \subseteq \mathscr{P}(B)\).

\end{thm}

\begin{proof}

Suppose \(\bigcup \mathcal{F} \subseteq B\). Let \(x\) be an arbitrary
element of \(\mathcal{F}\). Let \(y\) be an arbitrary element of \(x\).
Since \(y \in x\) and \(x \in \mathcal{F}\), by the definition of
\(\bigcup \mathcal{F}\), \(y \in \bigcup \mathcal{F}\). But then since
\(\bigcup \mathcal{F} \subseteq B\), \(y \in B\). Since \(y\) was an
arbitrary element of \(x\), we can conclude that \(x \subseteq B\), so
\(x \in \mathscr{P}(B)\). But \(x\) was an arbitrary element of
\(\mathcal{F}\), so this shows that
\(\mathcal{F} \subseteq \mathscr{P}(B)\), as required. \excl{~□}\qedhere

\end{proof}

\begin{nthm}{Theorem 3.4.7}
For every integer \(n\), \(6 \mid n\) iff \(2 \mid n\) and \(3 \mid n\).

\end{nthm}

\begin{proof}

Let \(n\) be an arbitrary integer.

(\(\to\)) Suppose \(6 \mid n\). Then we can choose an integer \(k\) such
that \(6k=n\). Therefore \(n = 6k = 2(3k)\), so \(2 \mid n\), and
similarly \(n = 6k = 3(2k)\), so \(3 \mid n\).

(\(\leftarrow\)) Suppose \(2 \mid n\) and \(3 \mid n\). Then we can
choose integers \(j\) and \(k\) such that \(n = 2j\) and \(n = 3k\).
Therefore \(6(j-k) = 6j - 6k = 3(2j) - 2(3k) = 3n - 2n = n\), so
\(6 \mid n\). \excl{~□}\qedhere

\end{proof}

\begin{mdsk}

\end{mdsk}

For the next exercise you will need the following definitions:

\bookmarksetup{startatroot}

\hypertarget{conclusions}{%
\chapter{Conclusions}\label{conclusions}}

\[
[x]_R = \{y \in A \mid yRx\}.
\] The set whose elements are all of these equivalence classes is called
\(A\) \emph{mod} \(R\). It is written \(A/R\), so \[
A/R = \{[x]_R \mid x \in A\}.
\] Note that \(A/R\) is a set whose elements are sets: for each
\(x \in A\), \([x]_R\) is a subset of \(A\), and \([x]_R \in A/R\).

\bookmarksetup{startatroot}

\hypertarget{works-cited}{%
\chapter{Works Cited}\label{works-cited}}

This work had been formatted and styled from the book \emph{How To Prove
It With Lean}, written by Daniel J. Velleman. \emph{How To Prove It With
Lean} contains short excerpts from \emph{How To Prove It: A Structured
Approach, 3rd Edition}, by Daniel J. Velleman and published by Cambridge
University Press.

\begin{Shaded}
\begin{Highlighting}[]
\KeywordTok{example}\NormalTok{ : square1 = square2 := }\KeywordTok{by} \KeywordTok{rfl}

\SpecialCharTok{++}\DocumentationTok{\#eval}\SpecialCharTok{::}\NormalTok{ square1 7     }\CommentTok{{-}{-}Answer: 49}
\end{Highlighting}
\end{Shaded}

\bookmarksetup{startatroot}

\hypertarget{additional-space}{%
\chapter{Additional space}\label{additional-space}}

Extra chapter to write more things if needed!!



\end{document}
