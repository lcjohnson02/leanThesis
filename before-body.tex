\newcommand{\fromtop}[1]{%
  \dimexpr-1in-\topskip-\topmargin-\headheight-\headsep+#1\relax
}
\newcommand{\fromleft}[1]{%
  \dimexpr-1in-\oddsidemargin+#1\relax
}
\setlength{\textwidth}{6in}
\begin{titlepage}
\cleardoublepage
\thispagestyle{empty}
\begin{textblock*}{\paperwidth}(\fromleft{0cm},\fromtop{144pt})
\centering
Automated Proof Verification with Lean
\end{textblock*}
\begin{textblock*}{\paperwidth}(\fromleft{0cm},\fromtop{300pt})
\centering
A Thesis Submitted to the Faculty of\\
\vspace{12pt}
Georgetown College\\
\vspace{12pt}
In Partial Fulfillment of the Requirements for the\\
\vspace{12pt}
Honors Program
\end{textblock*}
\begin{textblock*}{\paperwidth}(\fromleft{0cm},\fromtop{492pt})
\centering
By\\
\vspace{12pt}
Logan Johnson
\end{textblock*}
\begin{textblock*}{\paperwidth}(\fromleft{0cm},\fromtop{600pt})
\centering
Georgetown, Kentucky\\
\vspace{12pt}
May 2024
\end{textblock*}
\pagebreak
\thispagestyle{empty}
\begin{textblock*}{\paperwidth}(\fromleft{0cm},\fromtop{180pt})
\centering
Abstract\\
\vspace{12pt}
Automated Proof Verification with Lean\\
\vspace{12pt}
Logan C. Johnson\\
\vspace{12pt}
Director: Dr. Homer White, Ph.D.
\end{textblock*}
\begin{textblock*}{\textwidth}(\fromleft{1.5in},\fromtop{300pt})
\raggedright
This thesis is meant to explore Lean Theorem Prover and address its potential in automating the 
verification of mathematical proofs. I explore how lean functions as a functional programming 
language and how this changes the proof process. This leads to Lean treating theorems as functions 
that essentially convert known hypotheses into a new conclusion. Lean does this by taking advantage 
of typed objects rather than following a traditional set theory approach. I then directly compare 
the ways in which Lean proofs differ from traditional paragraph-style proofs and try to explain 
the upsides and downsides of these differences. Lean does end up being useful for automated proof 
verification, so long as one is able to learn the language and get used to the different style of 
proof. It is still being developed and is only becoming more powerful with time, so its userbase will 
likely only continue to grow.
\end{textblock*}
\pagebreak
\thispagestyle{empty}
\begin{textblock*}{\paperwidth}(\fromleft{2in},\fromtop{144pt})
APPROVED BY THE DIRECTOR OF HONORS THESIS:
\end{textblock*}
\begin{textblock*}{\paperwidth}(\fromleft{3in},\fromtop{192pt})
\underline{\hspace{4.5in}}
\end{textblock*}
\begin{textblock*}{\paperwidth}(\fromleft{3in},\fromtop{216pt})
Dr. Homer White, Department of Mathematics
\end{textblock*}
\begin{textblock*}{\paperwidth}(\fromleft{2in},\fromtop{280pt})
APPROVED BY SECOND READER OF HONORS THESIS:
\end{textblock*}
\begin{textblock*}{\paperwidth}(\fromleft{3in},\fromtop{328pt})
\underline{\hspace{4.5in}}
\end{textblock*}
\begin{textblock*}{\paperwidth}(\fromleft{3in},\fromtop{352pt})
Zachary May
\end{textblock*}
\begin{textblock*}{\paperwidth}(\fromleft{2in},\fromtop{396pt})
APPROVED BY THE HONORS PROGRAM:
\end{textblock*}
\begin{textblock*}{\paperwidth}(\fromleft{2in},\fromtop{444pt})
\underline{\hspace{4.5in}}
\end{textblock*}
\begin{textblock*}{\paperwidth}(\fromleft{2in},\fromtop{468pt})
Dr. Barbara Burch, Director
\end{textblock*}
\begin{textblock*}{\paperwidth}(\fromleft{1.5in},\fromtop{612pt})
DATE:\underline{\hspace{2in}}
\end{textblock*}
\end{titlepage}